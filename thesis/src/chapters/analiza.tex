\chapter{Analiza}\label{ch:analiza}

Analizom postojećih alata za upravljanje prevodima mogu se uočiti prednosti i mane tih alata. 
Zadatak analize postojećih sistema je upoznavanje sa problematikom kako bi se izbegle greške
koje je su ti sistemi napravili, ali i prikupljanje dobrih ideja i funkcionalnosti koje pružaju ti sistemi.


\section{Postojeća okruženja}\label{sec:analiza-postojeca_okruzenja}

Na tržištu postoji dosta rešenja koji se bave upravljanjem prevodima. Jedno od takvih rešenja je 
BabelEdit~\cite{BabelEdit}, editor prevoda za veb aplikacije. Program se instalira na klijentskom računaru.
Program podržava kreiranje projekta u koji se kasnije uvoze fajlovi sa prevodima. Kada korisnik završi 
sa prevođenjem, može da izveze prevode iz projekata u razne formate. Ovaj alat ne nudi mogućnost kolaboracije,
a fajlovi se nalaze na lokalnom računaru, i kao takav je pogodan jedino za manje projekte. Ovaj alat ima 
probni period od nekoliko dana, a kasnije se mora platiti.

Drugo popularno rešenje je Localazy~\cite{Localazy}, platforma koja podržava više od 50 radnih okvira, 
formata fajlova. Platforma se nalazi u oblaku i pristupa joj se preko veb pregledača. Na platformi se može 
napraviti novi projekat u okviru kojeg se mogu uvesti fajlovi sa prevodima za izabrani jezik. Istom projektu 
može pristupiti više prevodilaca i uređivati prevode. Kada se izabere fajl za prevod, sa leve strane se može 
videti kjuč za labelu koju treba prevesti, a pored njega i sam prevod. Prevodi su kasnije dostupni preko 
mreže za dostavu sadržaja (eng. content delivery network, CDN). Localazy omogućava besplatno korišćenje za 
projekte koji imaju najviše 200 ključeva, što odlikuje manje projekte.

Navedeni alati poseduju i integraciju sa sistemima za mašinsko prevođenje, što u mnogome olakšava posao 
prevodiocima. Svakako, na kraju je potrebna provera od strane čoveka kako bi sadržaj bio u većoj meri prilagođen. 


\section{Moguća unapređenja}\label{sec:analiza-moguca_unapredjenja}

Oba analizirana alata pružaju nekakav oblik obaveštenja da je prevodilac završio sa poslom. U slučaju BabelEdit
alata potrebno je da prevodilac izveze prevode i kasnije izvezene fajlove da pošalje programeru. Kod Localazy 
je proces umalo bolji jer prevodilac ne treba da šalje nikakve fajlove, već samo da obavesti programera, 
koji ce ih kasnije preuzeti sa CDN.

Ovde se može primetiti da je moguće automatizovati proces predaje prevoda. Potrebno je samo da klikom dugmeta
prevodilac obavesti sistem da je završio svoj posao. Ta akcija treba da pokrene izgradnju nove verzije aplikacije.
\chapter{Uvod}\label{ch:uvod}

Aplikacije koje su namenjene za globalno tržište treba pripremiti 
tako da korisnici koji pripadaju različitim govornim područijima i 
geografskim regionima razumeju sadržaj. Postupak dizajniranja aplikacije 
tako da podrži različite jezike se naziva internacionalizacija, a osobina 
aplikacije koja podržava više jezika se naziva višejezičnost. 

Proces prilagođavanja softvera za različite jezike može biti kompleksan. 
Veliku pomoć u tom procesu danas pružaju različita rešenja. Odabir rešenja 
uglavnom zavisi od odabira programskog jezika u kom će softver biti 
razvijan, kao i od korišćenja programskog okvira.

Sa druge strane, ne može se očekivati da programer zna sve jezike za koje je 
softver namenjen. To znači da se u proces internacionalizacije uključuju i 
prevodioci koji poznaju jezik korisnika. Oni implicitno 
postaju osobe koje razvijaju softver.

Pomenuta rešenja za olakšavanje internacionalizacije uglavnom su fokusirana
samo na tehničke probleme, odnosno probleme programera. Prevodi se čuvaju u
fajlovima i obično predstavljaju serijalizovanu strukturu mape, odnosno parove
"ključ -- vrednost". Takvi fajlovi su neretko teški za korišćenje od strane
netehnikičih lica, a prevodioci često nisu tehnička lica. Kako bi što bolje 
obavljali svoj posao, prevodiocima je potrebna neka vrsta alata koja im deluje
poznato, nešto na šta su navikli, odnosno nešto što koriste svaki dan. 
Prevodioce ne zanima koje rešenje su programeri izabrali, niti koji format 
serijalizacije se koristi. Njima je potreban familijaran interfejs za ažuriranje
tih fajlova.

Razvojem računarstva u oblaku, ljudi su navikli da im se sve nalazi u oblaku, 
odnosno da im je sve uvek dostupno, na svakom mestu i u bilo koje vreme. 
Paralelno sa rezvojem softvera u oblaku, popularnost stiču arhitekture zasnovane
na mikroservisima ali i razvoj aplikacija u kontejnerima. Za kontrolisanje 
velikog broja kontejnera se sve više korsti alat \textit{Kubernetes}. Pored toga, 
tendencija je da se hardver, odnosno infrastruktura, više ne održava u okviru
organizacije, već da se on iznajmljuje od dobavljača računarstva u oblaku 
(eng. \textit{cloud computing provider}). Ovakvim pristupom se omogućava lako 
prenošenje aplikacije sa jednog okruženja na drugo. Razvoj takvog softvera nosi
sa sobom dodatne izazove. 

U ovom radu biće opisan razvoj softvera u oblaku. Fokus će biti stavljen na 
arhitekturu mikroservisa, korišćenje alata \textit{Kubernetes} i proces pravljenja 
višejezične aplikacije. Za razvoj klijentskog, ali i serverskog dela 
kôda, koristiće se programski jezik \textit{JavaScript}. U svrhu ilustracije i
boljeg razumevanja biće razvijena aplikacija za upravljanje prevodima. 
Aplikacija je nazvana \textit{Interpres} i biće razvijana kao softver otvorenog kôda.

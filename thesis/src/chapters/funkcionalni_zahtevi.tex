\chapter{Funkcionalni zahtevi}\label{ch:funkcionalni_zahtevi}

Glavni zadatak sistema je da omogući lako upravljanje prevodima. To podrazumeva da prevodilac, koji nije tehničko 
lice, može sa lakoćom da obavlja svoj posao a da ne zna interne procese rada programerskog tima. S druge
strane, programerima treba omogućiti što veći stepen automatizacije. Jednostavno, treba razdvojiti što je
više moguće poslove prevodilaca i programera. Tu se mogu uočiti dve celine sistema, jedna celina koja podržava
proces rada prevodilaca, i druga celina koja podržava proces rada programera.


\section{Uređivač prevoda}\label{sec:funkcionalni_zahtevi-uredjivac_prevoda}

Uređivač prevoda predstavlja korisnički interfejs za uređivanje i obradu prevoda, a koristi ga prevodilac.
Kako se svaki prevod čuva u obliku "ključ -- vrednost", pored instance prevoda, potrebno je prikazati i sam 
ključ kako bi se mogla jednoznačno prikazati svaka instanca.

Ključ se može posmatrati i kao referenca na jednu grupu instanci prevoda koje predstavljaju istu nisku u različitim 
jezicima. Osim toga, on nema neku dodatnu vrednost za prevodioca.
Njima veću vrednost predstavlja instanca prevoda u jeziku koji je zajednički za prevodioca i programera. Na primer, kada 
programer želi da dobije prevode za srpski jezik, on će prevodiocu isporučiti instance prevoda na engleskom a prevodilac 
će uz odgovarajuće ključeve i instance prevoda na engleskom jeziku dodati prevode na srpskom jeziku. To znači da je na 
korisničkom interfejsu potrebno prikazati sve instance prevoda za određeni ključ.

Biblioteke za internacionalizaciju uglavnom pružaju nekakvu vrstu grupisanja ključeva. Ako se prevodi čuvaju u 
fajlovima, onda se jedan fajl može smatrati jednom grupom prevoda. Grupisanje je korisno jer se onda prevodi 
za određeni deo aplikacije obrađuju u istom kontekstu. Sa druge strane, ako se prevodi u aplikaciju dovlače 
preko mreže, inicijalno učitavanje aplikacije se može ubrzati jer nije potrebno učitati sve prevode, već se 
mogu učitati po potrebi. Na korisničkom interfejsu treba prikazati te grupe i omogućiti laku navigaciju kroz
grupe.

Pored grupisanja prevoda, sistem treba da pruža mogućnost pravljenja više projekata. Na primer, prevodilac 
može uređivati prevode za više aplikacija. Svaki projekat može sadržati ražličite skupove jezika na koje 
treba prevesti aplikaciju, ali i različita podešavanja za uvoz, odnosno izvoz prevoda.


\section{Uvoz i izvoz prevoda}\label{sec:funkcionalni_zahtevi-uvoz_izvoz}

Proces uvoza i izvoza prevoda iz sistema treba da bude što je više moguće automatizovan. Zapravo, integracija
sa sistemom za kontrolu verzija, kao što je \textit{GitHub}, bi pružala najudobniji rad. Integracija bi radila i za 
uvoz prevoda, ali i za izvoz. 

Prevodilac bi mogao preko dugmeta za sinhronizaciju da uveze nove promene 
koje su se pojavile u repozitorijumu. Promene koje mogu da se pojave u repozitorijumu su dodavanje novih 
ključeva ili brisanje postojećih. Za te potrebe, promene će se uvek preuzimati sa glavne grane iz sistema za 
kontrolu verzija. 

Izvoz prevoda bi isto mogao da se uradi jednim klikom na dugme. U tom slučaju sistem bi napravio novi zahtev 
za promenu (eng. \textit{Pull Request}) na \textit{GitHub}-u, koji bi zahtevao odobrenje od strane 
programera da se spoji sa glavnom granom koda.

Ovakvim razdvajanjem dva tima mogu raditi nesmetano. Prevodioci će uvek imati najsvežiju verziju prevoda, 
odnosno ključeva za labele koje treba prevesti i neće zavisiti od procesa rada programera. Sa druge strane,
programeri će na kraju dobijati zahteve za promene i mogu sami da odluče u kom trenutku žele da primene nove
prevode.


\section{Aplikacija u oblaku}\label{sec:funkcionalni_zahtevi-aplikacija_u_oblaku}

Aplikacije koje je potrebno prevesti na što veći broj jezika zahtevaju i timove prevodilaca. Kolaboraciju 
između više prevodilaca treba omogućiti preko mreže radi što veće udobnosti. Svaki prevodilac treba da ima 
mogućnost da pristupa aplikaciji, odnosno prevodima, sa svog računara. Kako bi se omogućila kolaboracija 
između više korisnika, potrebno je podatke skladištiti u oblaku. Veb aplikacije u oblaku pružaju najveću 
udobnost u tom smislu. Dovoljno je otvoriti aplikaciju u veb pregledaču i korisnik je spreman za rad.

Jedna od važnih osobina aplikacija na vebu je dostupnost. Dostupnost aplikacije predstavlja stepen u kojem je 
aplikacija operativna, funkcionalna i upotrebljiva za ispunjavanje zahteva korisnika. 
Računarstvo u oblaku pruža rešenja da aplikacija bude dostupna u skoro svakom trenutku.

Aplikacije u oblaku su podrazumevano javno dostupne preko veb adrese. Kako bi se podaci zaštitili,
potrebno je implementirati i sistem za autentikaciju. S obzirom da se već radi integracija sa \textit{GiHub}-om, radi
uvoza, odnosno izvoza prevoda, \textit{GitHub} se može iskoristiti kao provajder identiteta (eng. \textit{Identity Provider}).
To zači da je potrebno da prevodici naprave svoje naloge na \textit{GitHub}-u s kojima će se kasnije prijavljivati 
na sistem.
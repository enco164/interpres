\chapter{Zaključak}\label{ch:zakljucak}

Cilj ovog rada je da predloži i obrazloži informacioni sistem u oblaku koji bi pomogao 
programerima i prevodiocima tokom razvijanja višejezične aplikacije. Korišćenjem ovog 
sistema smanjuje se jaz između ove dve grupe koje zajedničkim snagama poboljšavaju 
korisničko iskustvo aplikacije na kojoj rade. Zbog sve prisutnije digitalizacije, 
ne sme se zaboraviti na grupe ljudi koje ne razumeju određeni jezik i zbog toga 
ne mogu da koriste aplikaciju.

Predloženo rešenje je u velikoj meri fleksibilno što se tiče podrške alata za prevođenje.
Bitno je da alat, za skladištenje prevoda, radi sa datotekama u formatu \textit{JSON}. 
U budućnosti se ostali formati mogu podržati jednostavnom implementacijom novog parsera
u mikroservisu \textit{integration} i poštovanjem interfejsa za komunikaciju sa 
mikroservisom \textit{core}. Sa druge strane, projekat \textit{Interpres} se previše 
oslanja na \textit{GitHub}. Glavne mane su nemogućnost uvoza i izvoza prevoda u fajlove 
i zavisnost od provajdera identiteta. Aplikacije koje je potrebno prevesti mogu koristiti 
neke druge sisteme za verzionisanje koda, a \textit{Interpres} to ne dozvoljava. Iz tog 
razloga bi uvoz i izvoz preko fajlova bio poželjniji. Isto tako, 
nije teško zamisliti da korisnici ne žele da prave naloge na \textit{GitHub}-u, već da 
žele da koriste neki drugi provajder identiteta kao što je \textit{Google},
\textit{Facebook}, ili pak žele da naprave nalog na samom sistemu.

Tema koja je ostala neobrađena, a koja je jako bitna, je autorizacija. U predloženom rešenju,
korisnik koji je autentifikovan ima pristup svim projektima. Šta više, svi korisnici imaju ista prava. 
Ovaj nedostatak predstavlja veliki rizik za bezbednost podataka. U narednoj verziji projekta trebalo bi osmisliti i 
implementirati prava pristupa. Korisnici bi se u tom slučaju dodavali u projekte sa određenim 
pravima, kao što su prava za čitanje, prava za menjanje sadržaja ili prava za podešavanje 
samog projekta.

Prilikom razvoja sistema korišćena je arhitektura mikroservisa. Ideja vodilja za odabir 
ove arhitekture je bila da informacioni sistem bude što je više moguće skalabilan, 
pouzdan i održiv. Sve veći trend korišćenja mikroservisa u oblaku za velike aplikacije 
potvrđuje da je to dobar izbor. 

Kako se računarstvo u oblaku razvija velikom brzinom, tako se razvijaju i novi alati 
koji pokušavaju da reše probleme koji nastaju na oblaku. S obzirom na sve veću upotrebu 
arhitekture mikroservisa, glavni problem predstavlja upravljanje većim brojem komponenti.
\textit{Kubernetes}, kao alat za rešavanje ovog problema, u potpunosti zadovoljava 
potrebe operacionog tima, ali i tima programera.

Alati za automatizaciju u mnogome pomažu razvoj softvera. Kontinuirano isporučivanje 
i kontinuirano raspoređivanje ubrzavaju ciklus razvoja i korisnici ranije mogu dobiti 
nove verzije softvera. Pored samih korisnika, benefit imaju i programeri jer mogu 
ranije da uoče probleme i brže da ih reše.

Potrebno je napomenuti da mikroservisi ne predstavljaju rešenje za sve. Iako se takvi 
sistemi lakše skaliraju i imaju bolju razdvojenost poslovne logike, sam razvoj ume da 
bude kompleksan. Prerano razmišljanje o optimizaciji i skaliranju sistema može usporiti 
razvoj i potencijalno dovesti do gašenja projekta jer funkcionalnosti sistema nisu 
isporučene na vreme. Sa druge strane, može se desiti da sam sistem uopšte nema potrebe 
da opsluži toliko korisnika. Onda se dobija kompleksan sistem bez prevelikog iskorišćenja,
što je potpuno suprotno od polazne pretpostavke. Potrebno je voditi se principom da se 
sistem optimizuje samo onda kada je to i zaista potrebno.

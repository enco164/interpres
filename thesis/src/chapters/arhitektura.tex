\chapter{Arhitektura i dizajn sistema}\label{ch:arhitektura}

Funkcionalni zahtev da aplikacija bude u oblaku, opisan u poglavlju 
\ref{sec:funkcionalni_zahtevi-aplikacija_u_oblaku}, nameće grubu sliku kako arhitektura sistema 
treba da izgleda. Aplikacije na Vebu su dominantno bazirane po modelu klijent--server arhitekture. 
Server u takvoj arhitekturi obezbeđuje uslugu klijentima koji je zahtevaju. Klijent--server 
arhitektura je višenamenska i modularna, a sa ciljem unapređenja upotrebljivosti, interoperabilnosti, 
fleksibilnosti i skalabilnosti. 

Veb aplikacije su nezavisne od platforme jer zahtevaju samo Veb pregledač, dok se kod standardne 
klijent--server arhitekture klijent mora instalirati na platformi korisnika. Pored toga, 
kod Veb aplikacija je protokol komunikacije definisan, koristi se \textit{HTTP}, dok se kod klijent--server
arhitekture protokol može izabrati po potrebi.

Komponente klijent -- server arhitekture, prilagođene za Veb, se mogu grupisati u tri kategorije: 
server, klijent i mreža. Uloga servera je da upravlja zajedničkim resursima, bazom podataka, da 
izvršava poslovnu logiku, kao i da kontroliše pristup i bezbednost podataka. Posao klijenta je da 
upravlja korisničkim interfejsom. Računarska mreža omogućava komunikaciju uzmeđu klijenta i servera, 
a komunikacija mora pratiti određene standarde.

\section{Arhitektura teških klijenata}\label{sec:arhitektura-spa}

Odlika arhitekture teških klijenata je da server šalje klijentu podatke i meta podatke, i prepušta 
mu da samostalno pripremi prezentaciju.~\cite{PVEB} Ovakvom podelom odgovornosti se smanjuje opterećenje 
servera, jer više ne mora da priprema prezentaciju. Arhitektura teških klijenata podrazumeva intenzivno 
korišćenje \textit{JavaScript}-a.

Ekstremni oblik ove arhitekture je arhitektura jedne stranice (eng. \textit{SPA - Single Page Application}).
Arhitektura jedne stranice propisuje da se čitava aplikacija učitava kroz jednu stranicu. Primenom 
ovih pravila se dobija potpuno odvojen klijent od servera. Klijent u ovom slučaju komunicira 
sa serverom u pozadini radi preuzimanja podataka i izvođenja transakcija. S obzirom da prezentacija
priprema na korisničkoj strani, korisnički interfejs ima bolji odziv, jer se ne prenosi kroz mrežu.

\section{Mikroservisi}\label{sec:arhitektura-mikroservisi}

Tradicionalan način razvijanja poslovnih aplikacija predstavlja razvoj takozvane monolitne arhitekture. 
Jedan monolit bi sadržao svu poslovnu logiku koju izvršava aplikacija. Kako aplikacija raste vremenom, 
tako raste i sam monolit, koji se sve teže održava. Problemi koji nastaju kod velikog monolita nisu 
samo problemi održavanja, već je i skaliranje upitno. 

Džejms Luis i Martin Fauler kažu da se izraz "Mikroservisna arhitektura" sve češće pojavlje poslednjih 
par godina kako bi opisao određeni način projektovanja softverskih aplikacija kao paketa usluga koji se 
mogu odvojeno isporučiti. Iako ne postoji precizna definicija ovog arhitektonskog stila, postoje određene 
zajedničke karakteristike u organizaciji oko poslovne logike, automatskog isporučivanja i decentralizovane 
kontrole podataka.~\cite{martinfowler_microservices} 

Mikroservisi predstavljaju aplikaciju koja je struktuirana kao kolekcija slabo vezanih servisa. Glavna 
ideja iza mikroservisne arhitekture je da se aplikacije lakše razvijaju i održavaju ako su podeljene 
na manje delove koji rade nesmetano zajedno.

Servisi se mogu posmatrati kao komponente sistema. Martin Fauler definiše komponente kao jedinice 
softvera koje se mogu nezavisno zameniti i unaprediti.~\cite{martinfowler_software_component}.

Biblioteke su komponente koje su povezane u program i pozivaju se pomoću funkcija koje se nalaze u 
memoriji, dok su servisi komponente koje su van procesa i koje komuniciraju preko mreže.
Servisi, naravno, mogu da koriste biblioteke. Glavni benefit ovakvog korišćenja servisa je da se mogu 
nezavisno isporučiti, što nije slučaj ako imamo jednu aplikaciju koja je sastavljena od skupa biblioteka.
Sa druge strane, daljinski pozivi preko mreže su skuplji, odnosno sporiji. 
Druga loša strana je da se logika teže prebacuje iz jednog servisa u drugi, odnosno refaktorisanje je 
otežano u tom smislu.

Da bi se monoliti lakše razvijali, programeri se uglavnom podele po tehnologiji, na primer u tim za 
bazu podataka, tim za poslovnu logiku i tim za korisnički interfejs. U arhitekturi mikroservisa je malo 
drugačije. Timovi se organizju oko jednog servisa i sastavljeni su od progamera različitih tehnologija.
To znači da su timovi nezavisni, kao i što su i sami servisi nezavisni, i organizovani su sa fokusom na 
poslovnu logiku, a ne na tehnologiju.

Jedna zgodna posledica deljenja monolita na servise je da servisi ne moraju da budu razvijeni u istim 
tehnologijama. Tako na primer, jedan servis može biti izgrađen u programskom jeziku \textit{C++} a drugi u 
\textit{NodeJS}, dokle god mogu da komuniciraju jedan sa drugim. Nekada je pogodnije koristiti drugi programski 
jezik jer je u njemu lakše rešiti problem.

Monolitne aplikacije preferiraju da imaju jednu bazu podataka za čuvanje podataka. Sa druge strane, kod 
mikroservisnih aplikacija se odluka o bazi podataka prepušta samom servisu. Neke probleme koje servis 
rešava je pogodnije rešiti relacionom bazom podataka, dok je za neki drugi servis možda pogodnija grafovska baza.
Iako i monolitne aplikacije mogu da koriste više tipova baza podataka, ova osobina je češća kod 
mikroservisa. Decentralizovana odgovornost za čuvanje podataka ima i svoje implikacije na ažuriranje 
podataka. Ovaj problem se može rešiti korišćenjem transakcija. Korišćenjem transakcija rešava se problem 
konzistentnosti, ali se smanjuje dostupnost. Zato, česta je odluka da se konzistentnost zameni 
odloženom konzistentnošću, u onim slučajevima gde je ona moguća. 

Kada se govori o mikroservisima često se sa razlogom postavlja pitanje postoji li razlika između arhitekture
mikroservisa i servisno orijentisane arhitekture (eng. \textit{Service Oriented Architecture}, \textit{SOA}). 
Glavne prednosti arhitekture mikroservisa su umnogome slične \textit{SOA}. Problem je taj što \textit{SOA} 
može da predstavlja mnogo različitih stvari, a najviše zbog prevelikog fokusa na kanal komunikacije 
(eng. \textit{Enterprise Service Bus}, \textit{ESB}) koji se koriste za integraciju monolitnih aplikacija. 
Može se reći da je arhitektura mikroservisa nastala iz stečenog iskustva razvijanja \textit{SOA}. 
Preuzeti su raznorazni dobri obrasci iz \textit{SOA}. \textit{ESB} je predstavljao preveliku kompleksnost 
pa je iz tog razloga zamenjen jednostavnim veb protokolima. Ljudi koji zagovaraju arhitekturu mikroservisa 
su iz ovih razloga krenuli da odbacuju naziv \textit{SOA}, dok drugi smatraju da su mikroservisi samo još 
jedan oblik \textit{SOA}. Kako bilo, činjenica da \textit{SOA} znači toliko različitih stvari, dragoceno 
je imati izraz koji preciznije opisuje ovu arhitekturu.

% CAP Teorema

\section{REST}\label{sec:arhitektura-rest}

Interoperabilnost je sposobnost da različiti sistemi rade zajedno. Da bi se ovo postiglo potrebni su 
standardi koje će sistemi poštovati. Kako u klijent -- server arhitekturi imamo dva različita sistema 
(klijent i server), oni moraju poštovati određene standarde za komunikaciju. Jedan od takvih standarda za 
komunikaciju je i \textit{REST} (eng. \textit{Representational State Transfer}). \textit{REST} je stil 
softverske arhitekture koji definiše skup pravila koja treba da se poštuju tokom pravljenja mrežnih 
aplikacija. Prvi put je predstavljen u doktorskoj disertaciji Roja Fildinga~\cite{REST_Roy}. Po \textit{REST}-u, 
fokus se prebacuje sa procedura na resurse (objekte). Za aplikaciju kažemo da je \textit{RESTful} ako poštuje pravila REST-a.


\textit{REST} definiše šest pravila koja moraju da se ispoštuju~\cite{REST_API}:

\begin{itemize}

	\item \textbf{Klijent -- server}: Razdvajanjem korisničkog interfejsa od servera briga o podacima 
	ostaje na serveru. Klijenti postaju lakši za portabilnost među raličitim sistemima, a server se može
    lakše skalirati.
	
    \item \textbf{Sistem bez stanja (eng. \textit{Stateless})}: Svaki zahtev sa klijenta ka serveru mora da sadrži 
	sve potrebne informacije da bi zahtev bio opslužen. Klijent se ne sme oslanjati na stanje servera. 
    Zapravo, stanje na serveru se zabranjuje. Iz tog razloga stanje sesije se u potpunosti čuva na klijentu.
	
    \item \textbf{Keširanje}: Server u svom odgovoru sa nekim podacima mora eksplicitno da navede da li 
    podaci mogu da se keširaju, ili ne. Ako server odgovori da su podaci kešabilni, klijent ih može 
    koristiti kasnije ako je zahtev za podacima isti. Server može napomenuti i vreme isteka keša.
    
    \item \textbf{Uniforman interfejs}: Uniformni interfejs definiše interfejs između klijenta i servera.
    On pojednostavljuje i razdvaja arhitekturu, što omogućava da se svaki deo razvija samostalno. 
    Četiri vodeća principa uniformnog interfejsa su:

    \begin{enumerate}
        \item \textbf{Zasnovanost na resursima}: Individualni resursi se mogu identifikovati u 
        zahtevima koristeći \textit{URI} kao identifikator resursa. Resursi kao takvi su konceptualno odvojeni 
        od reprezentacije koja se vraća klijentima. Na primer, server ne šalje svoju bazu podataka, 
        već radije, šalje \textit{HTML}, \textit{XML} ili \textit{JSON} koji predstavljaju traženi zapis iz baze podataka.

        \item \textbf{Upravljanje resursima kroz reprezentaciju}: Kada klijent sadrži reprezentaciju resursa,
        uključujući meta podatke u prilogu, onda ima dvoljno podataka da izmeni ili obriše resurse 
        sa servera, pod uslovom da ima ovlašćenje da to uradi.

        \item \textbf{Samoopisne poruke}: Svaka poruka sadrži dovoljno informacija da opiše na koji 
        način treba da bude obrađena. Na primer \textit{Internet media type} (nekada poznat pod nazivom \textit{MIME}) 
        može sadržati informaciju o tome koji parser treba da se pozove. Odgovori eksplicitno 
        označavaju da li imaju sposobnost keširanja.

        \item \textbf{Hipermedija kao pokretač aplikacije (\textit{HATEOAS})}: Klijenti isporučuju stanje kroz telo sadržaja,
        parametara upita, zaglavlja zahteva i traženog \textit{URI}-a (ime resursa). Servisi isporučuju stanje 
        kroz telo sadržaja, kodove u odgovoru i zaglavlje odgovora. To se tehnički naziva hipermedija 
        (hiperlinkovi unutar hiperteksta). \textit{HATEOAS} (eng. Hypermedia as the Engine of Application State) 
        takođe znači da, tamo gde je potrebno, veze su sadržane unutar vraćenog tela 
        (ili zaglavlja) odgovora, kako bi se isporučio URI za preuzimanje samog objekta ili srodnih 
        objekata. Uniformni interfejs koji svaki \textit{REST} servis mora pružiti je fundamentalna 
        osnova za njegov dizajn.
        
        
    \end{enumerate}
    
	\item \textbf{Slojevit sistem}: \textit{REST} omogućava korišćenje slojevitih sistema, gde klijent ne zna 
    eksplicitno da li komunicira sa krajnjim serverom ili sa posrednikom. Ovako se može povećati 
    skalabilnost uvođenjem balansera opterećenja ili uvođenjem keširanja na strani servera. 
	
	\item \textbf{Kod na zahtev (opciono)}: Serveri mogu obogatiti klijentsku stranu slanjem koda koji će biti izvršen na 
	klijentskoj strani. Ovo može uprostiti klijente jer se smanjuje broj funkcija koje je potrebno da 
    klijent ima. Primer na vebu bi bio slanje \textit{JavaScript} koda. 

\end{itemize}



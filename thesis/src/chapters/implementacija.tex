\chapter{Implementacija}\label{ch:impl}

Implementacija

- CI/CD (GitHub Actions) (Issue \#3)

- OpenAPI (Issue \#4)

- TypeORM (Issue \#6)
\section{Objektno-relaciono preslikavanje}\label{sec:objektno-relaciono-preslikavanje}

Dovoljno je reći da relacione baze podataka opstaju dugi niz godina kao veoma popularan mehanizam za skladištenje
podataka. Međutim, relacioni modeli se ne poklapaju najbolje sa objektnim modelima. Relacione baze podataka
predstavljaju podatke u tabelarnom formatu, dok se podaci kod objektno orijentisanih jezika mogu posmatrati kao
međusobno povezani grafovi objekata.

U objektnom modelu pristup podacima je potpuno drugačiji nego u relacionom modelu. Kod objektnog modela navigacija
izgleda tako što se uz pomoć asocijacija obilazi mreža objekata. Ovakav pristup nije efikasan za dohvatanje podataka iz
relacione baze. Obično se teži da se broj SQL upita smanji uz pomoć dohvatanja više entiteta odjednom, a potom započeti
obilazak mreže objekata, jer se umanjenjem broja potrebnih SQL upita za rekreiranje grafa objekata ubrzava rad
aplikacije.

Preslikavanje rezultata SQL upita u objekte se postize uz pomoć tehnike zvane objektno-relaciono preslikavanje
(eng. \textit{object-relational mapping}). Alati koji implementiraju tu tehniku se skraćeno zovu ORM alati. Uz pomoć ORM alata
tabele su predstavljene klasama a relacije između tih tabela su predstavljene relacijama tih klasa. Aplikacija treba da
radi sa klasama dok se ORM stara o fizičkim podacima unutar baze podataka. Odatle, aplikacija ne treba da se stara o
načinu povezivanja na bazu podataka, kako da izgradi SQL upit ili kako da koristi mehanizme zaključavanja da bi se
obezbedila konkurentnost. To je odgovornost ORM alata. Pored navedenog, uz pomoć ovih alata se lako može promeniti tip
baze podataka jer skoro svaki od njih pruža integracije za različite tipove baza, uglavnom je dovoljno samo promeniti
parametre za povezivanje na bazu.

Jedan od popularnijih ORM alata za JavaScript je TypeORM. On se može pokrenuti u različitim JavaScript platformama a
preporučeno je pisanje u TypeScript jeziku. Od obrazaca za projektovanje podržava aktivni zapis i preslikavanje podataka.
Autori ovog alata na svojoj internet stranici kažu da je TypeORM nastao pod velikim uticajem ostalih ORM alata, kao što
Hibernate, Doctrine i Entity Framework~\cite{TypeORM}.

Centralni objekat za preslikavanje tabele iz baze podataka u klasu programskog jezika je entitet (eng. \textit{Entity}).
Entitet se sastoji od kolona i relacija, i mora imati primarnu kolonu.

\begin{minted}[ fontsize=\footnotesize ]{TypeScript}
@Entity()
@Index(['projectId', 'key', 'lang'], { unique: true })
export class Translation {
  @PrimaryGeneratedColumn()
  id: number;
	
  @Column()
  projectId: number;
	
  @Column()
  key: string;
		
  @Column()
  lang: string;
	
  @Column()
  value: string;

  // ... ostatak implementacije
}
\end{minted}

\section{Kubernetes}\label{sec:kubernetes}

Do nedavno, većina softverskih aplikacija je razvijana kao veliki monoliti, koji su funkcionisali kao jedan proces 
ili kao mali broj procesa rasprostanjenih na više servera. Ovi zastareli sistemi su i dalje veoma rasprostranjeni. 
Njih karakteriše spor razvojni ciklus i ažuriraju se relativno retko. Na kraju svakog razvojnog ciklus, programeri
upakuju ceo sistem i predaju ga timu zaduženom za operacije, koji ga kasnije instalira i nadgleda. U slučaju hardverske
greške, tim za operacije ručno migrira sistem na preostale servere koji su bez greške.

Danas se ovako veliki monoliti rasčlanjuju na manje, nezavisne komponente koje se nazivaju mikroservisima. S obzirom
da su mikroservisi odvojeni jedni od drugih, mogu se razvijati, instalirati, ažurirati i skalirati svaki ponaosob.
Ovakva osobina omogućava češće promene na komponentama. S druge strane, povećanjem broja komponenti koje treba 
instalirati postaje sve teže konfigurisati, upravljati i očuvati ceo sistem u radnom stanju. Pored navedenog, mnogo 
je teže shvatiti kako i gde postaviti ove komponente kako bi se postigla veća iskorišćenost resursa, a samim tim i 
smanjiti cenu potrebnog hardvera. Odatle postoji potreba za automatizacijom, koja uključuje automatsku konfiguraciju,
nadzor  i rešavanje problema. Iz ovih razloga je razvijen Kubernetes.

Kubernetes omogućava programerima da sami instaliraju svoju aplikaciju, bez pomoći tima za operacije. Ali s druge 
strane, nemaju samo programeri benefit. Ovaj alat takođe pomaže operacionom timu tako što automatski nadgleda, 
i u slučaju greške pokreće nove instance aplikacija. To znači da se fokus operacionog tima preusmerava sa nadgledanja
pojedinačnih aplikacija na nadgledanje i upravljanje infrastrukture i Kubernetes alata, dok se Kubernetes stara o samim
aplikacijama.

Kubernetes apstrahuje hardversku infrastrukturu i pruža privid da je ceo data centar jedan veliki resurs. To omogućava
velikim kompanijama koje pružaju usluge računarstva u oblaku da ponude programerima jednostavnu platformu za pokretanje
raznih tipova aplikacija, a da pritom njihovi administratori sistema ne znaju koje su aplikacije pokrenute na njihovom
hardveru. Kako velike kompanije sve više prihvataju Kubernetes model kao jedan od boljih načina za pokretanje aplikacija,
tako Kubernetes postaje standardan model za računarstvo u oblaku~\cite{KIA}.

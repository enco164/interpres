\chapter{Implementacija}\label{ch:impl}

Implementacija

- CI/CD (GitHub Actions) (Issue \#3)

- OpenAPI (Issue \#4)

- TypeORM (Issue \#6)
\section{Objektno-relaciono preslikavanje}\label{sec:objektno-relaciono-preslikavanje}

Dovoljno je reći da relacione baze podataka opstaju dugi niz godina kao veoma popularan mehanizam za skladištenje
podataka. Međutim, relacioni modeli se ne poklapaju najbolje sa objektnim modelima. Relacione baze podataka
predstavljaju podatke u tabelarnom formatu, dok se podaci kod objektno orijentisanih jezika mogu posmatrati kao
međusobno povezani grafovi objekata.

U objektnom modelu pristup podacima je potpuno drugačiji nego u relacionom modelu. Kod objektnog modela navigacija
izgleda tako što se uz pomoć asocijacija obilazi mreža objekata. Ovakav pristup nije efikasan za dohvatanje podataka iz
relacione baze. Obično se teži da se broj SQL upita smanji uz pomoć dohvatanja više entiteta odjednom, a potom započeti
obilazak mreže objekata, jer se umanjenjem broja potrebnih SQL upita za rekreiranje grafa objekata ubrzava rad
aplikacije.

Preslikavanje rezultata SQL upita u objekte se postize uz pomoć tehnike zvane objektno-relaciono preslikavanje
(eng. \textit{object-relational mapping}). Alati koji implementiraju tu tehniku se skraćeno zovu ORM alati. Uz pomoć ORM alata
tabele su predstavljene klasama a relacije između tih tabela su predstavljene relacijama tih klasa. Aplikacija treba da
radi sa klasama dok se ORM stara o fizičkim podacima unutar baze podataka. Odatle, aplikacija ne treba da se stara o
načinu povezivanja na bazu podataka, kako da izgradi SQL upit ili kako da koristi mehanizme zaključavanja da bi se
obezbedila konkurentnost. To je odgovornost ORM alata. Pored navedenog, uz pomoć ovih alata se lako može promeniti tip
baze podataka jer skoro svaki od njih pruža integracije za različite tipove baza, uglavnom je dovoljno samo promeniti
parametre za povezivanje na bazu.

Jedan od popularnijih ORM alata za JavaScript je TypeORM. On se može pokrenuti u različitim JavaScript platformama a
preporučeno je pisanje u TypeScript jeziku. Od obrazaca za projektovanje podržava aktivni zapis i preslikavanje podataka.
Autori ovog alata na svojoj internet stranici kažu da je TypeORM nastao pod velikim uticajem ostalih ORM alata, kao što
Hibernate, Doctrine and Entity Framework~\cite{TypeORM}.

Centralni objekat za preslikavanje tabele iz baze podataka u klasu programskog jezika je entitet (eng. \textit{Entity}).
Entitet se sastoji od kolona i relacija, i mora da ima primarnu kolonu.

\begin{minted}{TypeScript}
export class Translation {
  @PrimaryGeneratedColumn()
  id: number;

  @Column()
  projectId: number;

  @Column()
  key: string;

  @Column()
  lang: string;

  @Column()
  value: string;

  // ... ostatak implementacije
}
\end{minted}


- Nest.js (Issue \#7)

- React

- Material UI

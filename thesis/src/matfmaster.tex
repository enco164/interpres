% !TeX root = ./matfmaster.tex
% Format teze zasnovan je na paketu memoir
% http://tug.ctan.org/macros/latex/contrib/memoir/memman.pdf ili
% http://texdoc.net/texmf-dist/doc/latex/memoir/memman.pdf
%
% Prilikom zadavanja klase memoir, navedenim opcijama se podešava
% veličina slova (12pt) i jednostrano štampanje (oneside).
% Ove parametre možete menjati samo ako pravite nezvanične verzije
% mastera za privatnu upotrebu (na primer, u b5 varijanti ima smisla
% smanjiti
\documentclass[12pt,oneside]{memoir}
\usepackage{minted}
\renewcommand\listingscaption{Primer koda}

% Paket koji definiše sve specifičnosti master rada Matematičkog fakulteta
\usepackage[latinica]{matfmaster}
%
% Podrazumevano pismo je ćirilica.
%   Ako koristite pdflatex, a ne xetex, sav latinički tekst na srpskom jeziku
%   treba biti okružen sa \lat{...} ili \begin{latinica}...\end{latinica}.
%
% Opicija [latinica]:
%   ako želite da pišete latiniciom, dodajte opciju "latinica" tj.
%   prethodni paket uključite pomoću: \usepackage[latinica]{matfmaster}.
%   Ako koristite pdflatex, a ne xetex, sav ćirilički tekst treba biti
%   okružen sa \cir{...} ili \begin{cirilica}...\end{cirilica}.
%
% Opcija [biblatex]:
%   ako želite da koristite reference na više jezika i umesto paketa
%   bibtex da koristite BibLaTeX/Biber, dodajte opciju "biblatex" tj.
%   prethodni paket uključite pomoću: \usepackage[biblatex]{matfmaster}
%
% Opcija [b5paper]:
%   ako želite da napravite verziju teze u manjem (b5) formatu, navedite
%   opciju "b5paper", tj. prethodni paket uključite pomoću:
%   \usepackage[b5paper]{matfmaster}. Tada ima smisla razmisliti o promeni
%   veličine slova (izmenom opcije 12pt na 11pt u \documentclass{memoir}).
%
% Naravno, opcije je moguće kombinovati.
% Npr. \usepackage[b5paper,biblatex]{matfmaster}

% Pomoćni paket koji generiše nasumičan tekst u kojem se javljaju sva slova
% azbuke (nema potrebe koristiti ovo u pravim disertacijama)
%\usepackage[latinica]{pangrami}

% Datoteka sa literaturom u BibTex tj. BibLaTeX/Biber formatu
\bib{matfmaster}

% Ime kandidata na srpskom jeziku (u odabranom pismu)
\autor{Uroš Milenković}
% Naslov teze na srpskom jeziku (u odabranom pismu)
\naslov{Razvoj veb aplikacije u oblaku za upravljanje prevodima u višejezičnim aplikacijama}
% Godina u kojoj je teza predana komisiji
\godina{2021}
% Ime i afilijacija mentora (u odabranom pismu)
\mentor{prof. dr Saša \textsc{Malkov}, vanredni profesor\\ Univerzitet u Beogradu, Matematički fakultet}
% Ime i afilijacija prvog člana komisije (u odabranom pismu)
\komisijaA{doc. dr Jelena \textsc{Graovac}, docent\\ Univerzitet u Beogradu, Matematički fakultet}
% Ime i afilijacija drugog člana komisije (u odabranom pismu)
\komisijaB{Anđelka \textsc{Zečević}, asistent\\ Univerzitet u Beogradu, Matematički fakultet}
% Ime i afilijacija trećeg člana komisije (opciono)
% \komisijaC{}
% Ime i afilijacija četvrtog člana komisije (opciono)
% \komisijaD{}
% Datum odbrane (odkomentarisati narednu liniju i upisati datum odbrane ako je poznat)
\datumodbrane{28. septembar 2021.}

% Apstrakt na srpskom jeziku (u odabranom pismu)
\apstr{%
Računarstvo u oblaku se svakog dana sve više razvija a prednosti softvera u oblaku 
su višestruke. Drugi trend koji se može primetiti u industriji je razvoj softvera u 
mikroservis arhitekturi. Iako ove dve tehnike imaju svoje prednosti, one sa sobom 
nose i određenu kompleksnost koja se ne sme zanemariti.

Bitna karakteristika modernih aplikacija je da budu višejezične. Takve aplikacije 
pružaju bolje korisničko iskustvo prevazilaženjem jezičkih barijera. Razvoj višejezičnih
aplikacija zahteva saradnju programera i prevodioca, koja se ogleda u tome da 
programeri šalju šifrarnike na jednom jeziku prevodiocu, a potom prevodilac vraća
popunjen šifrarnik na željenom jeziku. Šifrarnici se uglavnom prosleđuju u čoveku
čitljivom formatu za serijalizaciju, kao što su \emph{JSON}, \emph{YML} ili \emph{XML}.

Informacioni sistem koji pruža podršku saradnji programera i prevodioca, a pritom 
razvijen u oblaku, olakšao bi posao obema stranama u velikoj meri tako pto bi imao 
visoku dostupnost, standardizovao bi proces razmene prevoda i prevodiocima bi pružao
prijatno korisničko iskustvo.

Cilj ovog rada je da predloži i razvije informacioni sistem u oblaku koji bi pomogao
programerima i prevodiocima tokom razvijanja višejezične aplikacije. 

U okviru rada na tezi su obrazloženi, analizirani i diskutovani arhitektura i dizajn 
sistema koji podržavaju postavljene zahteve. Uz ovaj rad je napravljen sistem kao veb 
aplikacija u oblaku uz pomoć \emph{Kubernetes} sistema za automatsko skaliranje i 
upravljanje kontejnerskim aplikacijama.
}

% Ključne reči na srpskom jeziku (u odabranom pismu)
\kljucnereci{višejezičnost, računarstvo u oblaku, mikroservisi, javascript, react, nestjs, docker, kubernetes}

\begin{document}
% ==============================================================================
% Uvodni deo teze
\frontmatter
% ==============================================================================
% Naslovna strana
\naslovna
% Strana sa podacima o mentoru i članovima komisije
\komisija
% Strana sa posvetom (u odabranom pismu)
% \posveta{Mami, tati i dedi}
% Strana sa podacima o disertaciji na srpskom jeziku
\apstrakt
% Sadržaj teze
\tableofcontents*

% ==============================================================================
% Glavni deo teze
\mainmatter
% ==============================================================================

\chapter{Uvod}\label{ch:uvod}

Aplikacije koje su namenjene za globalno tržište treba pripremiti 
tako da korisnici koji pripadaju različitim govornim područijima i 
geografskim regionima razumeju sadržaj. Postupak dizajniranja aplikacije 
tako da podrži različite jezike se naziva internacionalizacija, a osobina 
da aplikacije koja podržava više jezika se naziva višejezičnost. 

Proces prilagođavanja softvera može biti kompleksan. Veliku pomoć u tom procesu
danas pružaju različita rešenja. Odabir rešenja uglavnom zavisi od odabira 
programskog jezika u kom će softver biti razvijan, kao i od korišćenja 
programskog okvira.

Sa druge strane, ne može se očekivati da programer zna sve jezike za koje je 
softver namenjen. To znači da se u proces internacionalizacije uključuju i 
prevodioci koji poznaju jezik na koji softver treba prevesti. Oni implicitno 
postaju osobe koje razvijaju softver.

Pomenuta rešenja za olakšavanje internacionalizacije uglavnom su fokusirana
samo na tehničke probleme, odnosno probleme programera. Prevodi se čuvaju u
fajlovima i obično predstavljaju serijalizovanu strukturu mape, odnosno parove
"ključ -- vrednost". Takvi fajlovi su neretko teški za korišćenje od strane
netehnikičih lica, a prevodioci često nisu tehnička lica. Kako bi što bolje 
obavljali svoj posao, prevodiocima je potrebna neka vrsta alata koja im deluje
poznato, nešto na šta su navikli, odnosno nešto što koriste svaki dan. 
Prevodioce ne zanima koje rešenje su programeri izabrali, niti koji format 
serijalizacije se koristi. Njima je potreban familijaran interfejs za ažuriranje
tih fajlova.

Razvojem računarstva u oblaku, ljudi su navikli da im se sve nalazi u oblaku, 
odnosno da im je sve uvek dostupno, na svakom mestu i u bilo koje vreme. 
Paralelno sa rezvojem softvera u oblaku, popularnost stiču arhitekture zasnovane
na mikroservisima ali i razvoj aplikacija u kontejnerima. Za kontrolisanje 
velikog broja kontejnera se sve više korsti alat Kubernetes. Pored toga, 
tendencija je da se hardver, ondnosno infrastruktura, više ne održava u okviru
organizacije, već da se on iznajmljuje od dobavljača računarstva u oblaku 
(eng. \textit{cloud computing provider}). Ovakvim pristupom se omogućava lako 
prenošenje aplikacije sa jednog okruženja na drugo. Razvoj takvog softvera nosi
sa sobom dodatne izazove. 

U ovom radu biće opisan razvoj softvera u oblaku. Fokus će biti stavljen na 
arhitekturu mikroservisa, korišćenje alata Kubernetes i proces pravljenja 
višejezične aplikacije. Za razvoj klijentskog, ali i serverskog dela 
kôda, koristiće se programski jezik JavaScript. U svrhu ilustracije i
boljeg razumevanja biće razvijena aplikacija za upravljanje prevodima. 
Aplikacija je nazvana Interpres i biće razvijana kao softver otvorenog kôda.

\chapter{Funkcionalni zahtevi}\label{ch:funkcionalni_zahtevi}

Funkcionalni zahtevi

\chapter{Slučajevi upotrebe aplikacije Interpres}\label{ch:slucajevi_upotrebe}


\renewcommand{\labelenumii}{\arabic{enumi}.\arabic{enumii}}
\renewcommand{\labelenumiii}{\arabic{enumi}.\arabic{enumii}.\arabic{enumiii}}
\renewcommand{\labelenumiv}{\arabic{enumi}.\arabic{enumii}.\arabic{enumiii}.\arabic{enumiv}}

Razvoj softvera iterativnom metodom diktira i uređivanje prevoda za aplikaciju. Na kraju svakog 
ciklusa, programeri obaveste prevodioce da su završili kako bi prevodioci mogli da započnu svoj 
ciklus prevođenja. 

U ovom poglavlju će biti opisani slučajevi upotrebe. Na dijagramu \ref{fig:slucajevi_upotrebe} je 
prikazana šira slika svih slučajeva upotrebe. 

\begin{figure}[H]
    \centering
    \includegraphics[width=1\textwidth]{slucajevi_upotrebe}
    \caption{BPMN dijagram slučajeva upotrebe}
    \label{fig:slucajevi_upotrebe}
\end{figure}

%%%%%%%%%%%%%%%%%%%%%%
\section{Prijava korisnika uz pomoć GitHub IdP}

\textbf{Kratak opis}: Korisnik se prijavljuje sa sopstvenim kredencijalima

\textbf{Učesnici}: Korisnik

\textbf{Postuslov}: Korisnik je prijavljen

\textbf{Glavni tok}:
\begin{enumerate}
    \item Korisnik klikne na dugme za prijavu
    \item Sistem šalje zahtev prema \textit{Gateway}-u za prijavu
    \item \textit{Gateway} vraća preusmeravanje (302 \textit{Redirect}) na stranicu za prijavu 
    provajdera identiteta (u ovom slučaju \textit{GitHub})
    \item Šalje se zahtev za dobijanje stranice za prijavu provajdera 
    identiteta
    \item \textit{GitHub} šalje stranicu za prijavu korisnika
    \item Korisnik unosi svoje kredencijale
    \item Korisnik šalje zahtev za prijavu klikom na dugme
    \item \textit{GitHub} dobija podatke za prijavu i nakon uspešnog prijavljivanja 
    šalje kod za pristup veb aplikaciji
    \item Veb aplikacija prosleđuje \textit{Gateway}-u kod za pristup
    \item \textit{Gateway}, uz zahtev za dobijanje podataka o korisniku, prosleđuje 
    kod za pristup \textit{GitHub}-u 
    \item \textit{GitHub Gateway}-u vraća podatke o korisniku
    \item Na osnovu dobijenih podataka \textit{Gateway} generiše \textit{JWT} token
    \item \textit{Gateway} šalje \textit{JWT} token veb aplikaciji
    \item Veb aplikacija omogućava prikaz zaštićene stranice korisniku 
    i čuva \textit{JWT} token za buduće komunikacije
\end{enumerate}

\begin{figure}[H]
    \centering
    \includegraphics[width=1\textwidth]{prijava_korisnika}
    \caption{Dijagram sekvence -- prijava korisnika}
\end{figure}


%%%%%%%%%%%%%%%%%%%%%%
\section{Pravljenje projekta}

\textbf{Kratak opis}: Prevodilac ima mogućnost da napravi novi projekat

\textbf{Učesnici}: Prevodilac

\textbf{Postuslov}: Projekat je napravljen

\textbf{Glavni tok}:
\begin{enumerate}
    \item Sistem prikazuje opciju za pravljenje novog projekta
    \item Prevodilac bira akciju za pravljenje novog projekta
    \item Sistem prikazuje formu sa tekstualnim poljima za unos podataka o projektu
    \item Prevodilac unosi 
    \begin{itemize}
        \item naziv projekta
        \item vlasnika \textit{GitHub} repozitorijuma
        \item \textit{GitHub} repozitorijum
        \item putanju na kojoj će se čuvati fajlovi sa prevodima i
        \item skup jezika za koje su potrebni prevodi.
    \end{itemize}
    \item Prevodilac klikne na dugme za testiranje konekcije
    \begin{itemize}
        \item Ukoliko su podešavanja pravilno unešena i \textit{GitHub} aplikacija uspešno instalirana u repozitorijum,
        sistem prikazuje uspešnu poruku zajedno sa jezicima koje je pronašao u repozitorijumu
        \item Ukoliko podešavanja nisu pravilno unešena ili \textit{GitHub} aplikacija nije uspešno instalirana u repozitorijum,
        sistem prikazuje poruku sa greškom.
    \end{itemize}
    \item Prevodilac klikne na dugme za pravljenje projekta
    \item Sistem čuva projekat
    \item Sistem prevodiocu vraća stranicu sa podacima o projektu
\end{enumerate}

\begin{figure}[H]
    \centering
    \includegraphics[width=1\textwidth]{create_project}
    \caption{Korisnički interfejs -- pravljenje projekta}
\end{figure}


%%%%%%%%%%%%%%%%%%%%%%
\section{Uvoz prevoda}

\textbf{Kratak opis}: Prevodilac želi da uveze prevode sa \textit{GitHub}-a

\textbf{Učesnici}: Prevodilac

\textbf{Postuslov}: Prevodilac se nalazi na projektu u koji želi da uveze prevode

\textbf{Glavni tok}:
\begin{enumerate}
    \item Sistem prikazuje opcije za uvoz i izvoz prevoda
    \item Prevodilac bira akciju za uvoz prevoda
    \item Sistem prikazuje modal sa porukom da će se uvozom prevoda sve 
    izmene koje nisu izvezene izgubiti
    \item Prevodilac klikne na dugme "\textit{OK}" ukoliko nema prevoda koji nisu sačuvani
    \item Sistem uvozi prevode
    \item Sistem zatvara modal
\end{enumerate}

\textbf{Alternativni tok}: Ukoliko prevodilac u koraku 4. klikne na dugme "\textit{Cancel}", 
sistem zatvara modal i prevodi neće biti uvezeni.


%%%%%%%%%%%%%%%%%%%%%%
\section{Izmena prevoda}

\textbf{Kratak opis}: Prevodilac vrši izmene kako bi dodao novi prevod ili izmenio postojeći

\textbf{Učesnici}: Prevodilac

\textbf{Postuslov}: Uspešno izvršena izmena prevoda

\textbf{Glavni tok}:
\begin{enumerate}
    \item Sistem prikazuje listu prevoda sa odgovarajućim ključevima
    \item Prevodilac klikne na željeni ključ za koji želi da unese ili izmeni prevod
    \item Sistem vraća tekstualna polja za unos za onoliko jezika koliko je u tom 
    trenutku dostupno za izabrani ključ
    \item Prevodilac unosi ili menja jedan ili više prevoda
    \item Prevod se automatski šalje na čuvanje u trenutku kada tekstualno polje izgubi fokus
    \item Sistem čuva izmenjeni prevod
\end{enumerate}

\begin{figure}[H]
    \centering
    \includegraphics[width=1\textwidth]{translation_editor}
    \caption{Korisnički interfejs uređivača prevoda}
\end{figure}


%%%%%%%%%%%%%%%%%%%%%%
\section{Izvoz prevoda}

\textbf{Kratak opis}: Prevodilac želi da izveze prevode izabranog projekta i 
na taj način napravi zahtev za promenu na \textit{GitHub}-u

\textbf{Učesnici}: Prevodilac

\textbf{Preduslov}: Prevodilac se nalazi na projektu koji želi da izveze

\textbf{Postuslov}: Uspešno izvezen sadržaj prevoda

\textbf{Glavni tok}:
\begin{enumerate}
    \item Sistem prikazuje opcije za uvoz i izvoz prevoda
    \item Prevodilac bira akciju za izvoz prevoda
    \item Sistem prikazuje modal sa formom za unos naslova i opisa 
    zahteva za promenu
    \item Prevodilac unosi naslov i opis u tekstualna polja
    \item Korisnik klikne na dugme \textit{OK}
    \item Sistem pokreće akciju izvoza prevoda
    \item Na \textit{GitHub}-u se pravi zahtev za promenu sa izmenjenim fajlovima za prevode
\end{enumerate}

\begin{figure}[H]
    \centering
    \includegraphics[width=0.7\textwidth]{izvoz_prevoda}
    \caption{Dijagram aktivnosti -- izvoz prevoda}
\end{figure}

\begin{figure}[H]
    \centering
    \includegraphics[width=1\textwidth]{export_translations}
    \caption{Korisnički interfejs za izvoz prevoda}
\end{figure}


\chapter{Implementacija}\label{ch:impl}

Implementacija prati opisanu arhitekturu iz prethodnih poglavlja. 
Korisnički interfejs je zasnovan na arhitekturi jedne stranice, 
uz pomoć okruženja \textit{React}. Komunikacija između klijenta i servera 
je preko \textit{REST API}-a. Server je implementiran u arhitekturi
mikroservisa uz pomoć \textit{NestJS}. Za bazu podataka 
je izabrana \textit{PostgreSQL}. Za potrebe pravljenja zahteva za promenu 
na kodu, server komunicira preko \textit{GitHub} aplikacije. Autentikacija 
je implementirana preko \textit{OAuth} protokola, a za provajdera 
identiteta je izabran \textit{GitHub IdP}. Kao alat za \textit{CI/CD} se koriste 
\textit{GitHub Akcije}. 

Implementirano rešenje je nazvano \textit{Interpres}. Aplikacija je 
višejezična, odnosno može prikazati korisnički interfejs u više 
jezika. Za potrebe uređivanja prevoda je korišćen upravo \textit{Interpres}.

Izvorni kod je otvoren i može se naći na \url{https://github.com/enco164/interpres}.


\section{Komponente sistema}

Na slici \ref{fig:komponente} su prikazane komponente sistema kao i
njihove zavisnosti. Važno je napomenuti da, iako mikroservisi \textit{core} i 
\textit{user-management} koriste različite baze podataka,
obe baze se zapravo nalaze u istom sistemu za upravljanje bazama podataka. 
Ova odluka je donešena radi boljeg iskorišćenja resursa. 

\begin{figure}[h]
  \centering
  \includegraphics[width=0.7\textwidth]{komponente}
  \caption{Komponente sistema}
  \label{fig:komponente}
\end{figure}

U nastavku će biti opisana svaka komponenta, kao i njena 
zavisnost sa drugim komponentama.

\subsection{SPA}
Komponenta \textit{SPA} predstavlja klijentsku stranu aplikacije, implementiranu 
u stilu arhitekture jedne stranicne. Izgrađena je 
uz pomoć razvojnog okruženja \textit{React}. Za stilizovanje korisničkog 
interfejsa korišćena je biblioteka \textit{Material UI}. 

Kada se pokrene izgradnja \textit{React} projekta, artifakti koji se dobiju 
su \textit{HTML}, \textit{JavaScript} i \textit{CSS} fajlovi. Da bi korisnik dobio fajlove, 
potreban je veb server, i u te svrhe je izabran \textit{Nginx}.

Komponenta \textit{SPA} dobija podatke sa servera. Komunikacija sa serverom 
je preko protokola \textit{REST}, a ulazna tačka je mikroservis \textit{gateway}.

\subsection{Gateway}
Komponenta \textit{gateway} jedina "otvara kapiju" ka spoljnom svetu.
Ona služi da prihvati zahteve sa klijentske strane i prosledi ih 
drugim mikroservisima. S obzirom da ona predstavlja svojevrsnu "kapiju",
autentikacija je implementirana baš tu.

Provera pristupa se radi preko \textit{GitHub IdP}. Ako se korisnik 
prvi put prijavljuje na sistem, njegovi podaci o imenu i prezimenu će 
biti poslati mikroservisu \textit{user-management}. 

Komunikacija sa ostalim mikroservisima (\textit{core} i \textit{user-management})
se odvija preko protokola \textit{TCP} podržanog od strane \textit{NestJS}.

\subsection{User Management}
Cilj ove komponente je da se brine o podacima korisnika. Pošto se podaci o korisnicima 
preuzimaju od eksternih servisa, u ovom slučaju od \textit{GitHub}-a, kao dobra praksa se 
pokazalo da ne treba zavisiti od ključeva eksternih sistema. 
Ovde se konkretno misli na jedinstveni identifikator korisnika. Kao 
poboljšanje sistema, mogao bi da se implementira pristup sistemu preko 
nekog drugog provajdera identiteta. U tom slučaju može doći do kolizije 
ključeva, odnosno ne možemo da budemo sigurni da će različiti provajderi identiteta 
davati različite ključeve.

Pored brige o primarnim ključevima za korisnike, ovaj mikroservis služi 
i kao svojevrsna optimizacija. Naime, mnogo je brže kontaktirati 
mikroservis koji je u \textit{Kubernetes} klasteru nego neki eksterni 
servis.

\subsection{Core}
Mikroservis \textit{core} predstavlja srž aplikacije. On se bavi čuvanjem 
podataka o prevodima i o projektima. Tu se nalazi i poslovna logika za 
grupisanje i preslikavanje prevoda u strukturu koja je pogodna za 
klijentsku stranu aplikacije ili za mikroservis \textit{integration}.

\subsection{Integration}
Ova komponenta ima funkciju integracije sa sistemom za verzionisanje koda. 
Na zahtev mikroservisa \textit{core}, ona može preko \textit{GitHub Aplikacije}, 
da dohvati prevode sa repozitorijuma i da napravi zahtev za promenu sa 
novim izmenama. Ako bi se u budućnosti implementirala integracija sa nekim 
drugim sistemom za verzionisanje koda, ovaj mikroservis je pravo mesto za to.

\subsection{GitHub Application i GitHub IdP}
Da bi se napravila integracija sa \textit{GitHub}-om, za potrebe menjanja 
koda, potrebno je napraviti \textit{GitHub Application}. Aplikacija 
ustvari daje samo pristupne ključeve, a na programeru je dalje da implementira 
i postavi aplikaciju na neki server. Implementacija aplikacije u \textit{NodeJS} 
je preko \textit{GitHub}-ove biblioteke, nazvane \textit{octokit}. Preko ove 
biblioteke se dobija interfejs za sve akcije koje je moguće uraditi u 
repozitorijumu. Dve glavne koje su implementirane su čitanje fajlova sa 
prevodima i pravljenje zahteva za promenu, i koriste se za uvoz i za 
izvoz prevoda.

\textit{GitHub IdP} je eksterna komponenta i služi kao provajder identiteta. 
Slično kao i za \textit{GitHub Application}, potrebna je samo registracija 
za dobijanje pristupnih ključeva kojima se pristupa interfejsu provajdera. 
Za implementaciju pristupa korićena je biblioteka \textit{Passport.js}, 
preporučena od strane \textit{NestJS}, koja ima implementaciju 
protokola \textit{OAuth2}, koji koristi \textit{GitHub IdP}.

\section{Klijent}
Klijentski deo je izgrađen uz pomoć radnog okvira \textit{React}, a početna 
organizacija koda uz pomoć \textit{create-react-app}, preporučenog alata za 
generisanje \textit{React} projekta od strane \textit{Facebook}-a. 
Projekat je izgenerisan na jeziku \textit{Typescript}. 

Organizacija projekata je podeljena po funkcionalnostima, i one su 
\textit{auth} (odgovorna za autorizaciju), \textit{projects} (odgovorna za 
podešavanje projekta), \textit{import-export} (odgovorna za 
uvoz, odnosno izvoz prevoda) i \textit{translations} (odgovorna za uređivanje 
prevoda).

Sve \textit{React} komponente su pisane kao funkcijske komponente 
(eng. \textit{function components}) uz korišćenje kuka (eng. \textit{hooks}).
Programeri koji po prvi put koriste \textit{React} uglavnom nalaze da je 
ovako napisan kod nečitljiv jer deluje da je pomešana poslovna logika komponente 
sa prikazom korisničkog interfejsa. Ustvari, mišljenje da je poslovna logika 
pomešana sa prikazom korisničkog interfejsa nastaje iz toga da u zvaničnoj 
dokumentaciji nije predložena arhitektura, već se samo opisuje tehnologija.
Programeru je ostavljeno na razmišljanje kako da organizuje svoju aplikaciju.
Prilikom razvoja projekta poštovan je princip da se sva poslovna logika 
nalazi u kuki, a da funkcijska komponenta koristi kuku i prikazuje sadržaj. 
Ovim principom se postiže veća čitljivost, razdvajaju se odgovornosti, lakše se testira 
automatskim testovima, a samim tim se i kod lakše održava. 
Ovaj princip je prikazan na primeru koda \ref{code:primer}.

\begin{listing}[h]
  \centering
  \begin{minted}[ fontsize=\footnotesize ]{jsx}
  // kuka: usePrimer.ts 
  export const usePrimer = () => {
    const [count, setCount] = useState(0);

    return {
      count,
      handleClick: () => setCount(count + 1),
    };
  }

  // komponenta: primer.tsx 
  export const Primer = () => {
    const {count, handleClick} = usePrimer();

    return (
      <div>
        <p>Kliknuli ste {count} puta</p>
        <button onClick={handleClick}>
          Klikni me
        </button>
      </div>
    );
  }
  \end{minted}
\caption{Princip pisanja funkcijskih komponenata sa kukama}
\label{code:primer}
\end{listing}

Za upravljanje stanjem u aplikaciji korišćen je \textit{redux}. On radi 
po principu centralizovanog skladišta, što znači da njemu pristupaju sve 
komponente. Na taj način je napravljena sigurnost da će podaci biti konzistentni 
i da neće biti particionisani. Ako dve komponente zahtevaju isti podatak, one ga neće čuvati u svom stanju 
već će ga potraživati sa istog mesta.

\section{Server}
Svaki mikroservis je napravljen kao zasebna aplikacija i generisan uz pomoć 
\textit{NestJS} interfejsa za komandnu liniju. Ako neki mikroservis treba 
da zna za postojanje nekog drugog, informacija o lokaciji će biti prosleđenja 
kroz konfiguraciju, odnosno kroz sistemske promenljive. U konfiguraciji 
se pored toga čuvaju pristupni ključevi za \textit{GitHub Application}, 
\textit{GiHub IdP}, i lokacija i kredencijali za bazu podataka. Konfiguracija 
mikroservisa se učitava pri svakom podizanju.

Mikroservisi koji čuvaju stanje u bazi podataka su \textit{core} i 
\textit{user-management}. Kako \textit{NestJS} koristi u pozadini \textit{TypeORM}, 
iskorišćena je njegova funkcija migriranja baze podataka. Migracije služe 
za promenu sheme baze podataka u produkcionom okruženju. One osiguravaju da 
će prebacivanje na novu verziju sheme biti sigurno i da neće doći do gubitka 
podataka. Migracioni fajl sadrži klasu koja implementira 
\texttt{MigrationInterface}, a potrebno je implementirati dve metode: 
\texttt{up} i \texttt{down}. Prva služi da se baza migrira na višu 
verziju, a druga služi ako je u nekom slučaju potreban povratak na prethodnu. 
Unutar tih metoda treba napisati \textit{SQL} naredbe za migracije. Pored 
ručnog pisanja migracija, \textit{TypeORM} pruža i mogućnost generisanja 
migracionih klasa, jer može izračunati prethodni oblik modela, a odatle i 
razliku koju treba primeniti na bazu podataka kako bi podržala novi model. 

Stil pisanja koda na serveru je reaktivan uz \textit{rx.js} biblioteku koja
implementira obrazac "posmatrač". Server napisan u stilu mikroservisa,
a komunikacija među servisima je sinhrona, odnosno po principu 
"zahtev -- odgovor". To znači da dok neki servis čeka na odgovor drugog 
servisa, prvi ostaje blokiran dok ne dobije odgorvor. Kako se programi napisani 
u \textit{JavaScript}-u izvršavaju u jednoj niti ovo postaje veliki problem. 
Iz tog razloga \textit{NestJS} pruža implementaciju \textit{HTTP} klijenta 
koji prima odgovore asinhrono. On će poslati \textit{HTTP} zahtev koji će biti razrešen kada 
stigne odgovor. Na taj način mikroservis koji je poslao \textit{HTTP} zahtev 
može obavljati i neki drugi posao dok odgovor ne stigne. Korišćenjem obrasca 
posmatrač se ova asinhronost lakše apstrahuje. Potrebno je napraviti zahtev 
i onda se pretplatiti na odgovor. Dok se čeka odgovor, pogram je slobodan da 
obavlja neki drugi posao. Kada odgovor stigne biće pozvana funkcija 
za obradu pretplate i tok izvršavanja će se nastaviti.
Pisanje reaktivnog koda je umnogome olakšano jer je i sam \textit{NestJS} 
napisan uz pomoć \textit{rx.js}. Na primeru koda \ref{code:rxjs} je prikazano
korišćenje \textit{rx.js} na metodi uvoza prevoda.


\begin{listing}[h]
  \centering
  \begin{minted}[ fontsize=\footnotesize ]{js}
importProject({ projectId }: ImportRequest) {
  return this.projectRepository.getProjectById(projectId)
    .pipe(
      throwIfEmpty(
        () => new NotFoundException(`Project with id ${projectId} not found`)
      ),
      concatMap((project) =>
        forkJoin([
          this.integrationMicroserviceClientSend(
            "import",
            {
              owner: project.githubOwner,
              repo: project.githubRepo,
              translationsLoadPath: project.lngLoadPath,
            }
          ),
          this.translationRepository.remove(project.translations)
        ])
      ),
      concatMap(([dataFromGithub, ]) =>
        this.importParsedTranslations(dataFromGithub, projectId)
      )
    );
}  
  \end{minted}
\caption{Metoda za uvoz prevoda napisana u reaktivnom stilu}
\label{code:rxjs}
\end{listing}


\section{Automatizacija}
Kontinualna integracija, isporučivanje i raspoređivanje je implementirano uz 
pomoć \textit{GitHub}-a. Na glavnoj (\textit{"main"}) grani 
je postavljeno pravilo zaštite, odnosno onemogućeno je direkto slanje 
koda na tu granu. Za svaku izmenu koda potrebno otvoriti zahtev za 
promenu. Pored toga, postavljeno je pravilo da, grana koju treba spojiti 
na glavnu granu, mora sadržati sve izmene koje se nalaze na glavnoj grani.

Na događaj otvaranja novog zahteva za izmenu koda, pokreće se niz 
\textit{GitHub} akcija. Za klijentski deo i za svaki mikroservis 
se pokreće naredba izgradnje i naredba testiranja. Uz navedena 
ograničenja za spajanje grana, i sa ovom \textit{GitHub} akcijom, 
osigurava se da će spojeni kod biti istestiran pre spajanja. To znači da 
na glavnoj grani ne bi trebalo da se pojavi neka greška koja bi inače mogla 
da se otkrije testiranjem.

Kao i za zahtev za izmenu koda, kada se grana spoji u glavnu granu 
postoji niz \textit{GitHub} akcija, za svaku komponentu po jedna akcija.
Tu se izgrade \textit{Docker} slike koje se kasnije isporučuju na 
\textit{Docker} javni registar. Potom, kada su sve slike isporučene na 
\textit{Docker} registar, pokreće se \textit{GitHub} akcija koja započinje 
raspoređivanje na \textit{Kubernetes}.

Na primeru koda \ref{code:cicd} je prikazan tok kontinualne isporuke
mikroservisa \textit{gateway} na \textit{Docker} javni registar.

\begin{listing}[h]
  \centering
  \begin{minted}[ fontsize=\footnotesize ]{yaml}
name: Build & Publish API Gateway docker image

on:
  push:
    branches:
      - main

jobs:
  build-docker-and-publish:
    runs-on: ubuntu-latest
    environment: main

    steps:
      - name: Checkout code
        uses: actions/checkout@v2

      - name: Use Node.js 14
        uses: actions/setup-node@v1
        with:
          node-version: '14'

      - name: Determine Docker Tag
        run: echo "DOCKER_TAG=${{ github.sha }}"  >> $GITHUB_ENV
      - run: |-
          echo DOCKER_TAG:  $DOCKER_TAG

      - name: DockerHub login
        env:
          DOCKERHUB_USERNAME: ${{ secrets.DOCKERHUB_USERNAME }}
          DOCKERHUB_PASSWORD: ${{ secrets.DOCKERHUB_PASSWORD }}
        run: |
          docker login -u $DOCKERHUB_USERNAME -p $DOCKERHUB_PASSWORD

      - name: Build docker image
        run: |
          docker build \
            -t interpres/api-gateway:latest \
            -t interpres/api-gateway:$DOCKER_TAG \
            ./server/api-gateway

      - name: Push docker image
        run: |
          docker push --all-tags interpres/api-gateway
\end{minted}
\caption{Tok kontinualne isporuke mikroservisa \textit{gateway}}
\label{code:cicd}
\end{listing}

\section{Produkciono okruženje}
Razne kompanije pružaju usluge iznajmljivanja računara u oblaku. Neki od 
poznatijih proizvoda su \textit{Amazon Web Services}, \textit{Microsoft Azure}
i \textit{Google Cloud Platform}. Sa druge strane, \textit{Kubernetes} je 
nezavisan od platforme. Iako ga je moguće instalirati i pokrenuti na sopstvenom 
serveru, taj posao je mukotrpan pa je pametnije izabrati neki proizvod gde se 
\textit{Kubernetes} može podići uz par klikova. 

Kako je \textit{Google} razvio \textit{Kubernetes}, pretpostavka je da je 
na \textit{Google Cloud Platform} uvek malo prednjači, pa je iz tog razloga 
on izabran za aplikaciju \textit{Interpres}.

Podešavanje na \textit{Google Cloud Platform} je jednostavno. Preko 
korisničkog interfejsa je potrebno napraviti klaster. Za klaster je 
potrebno izabrati tip virtualne mašine, kao i lokaciju servera na kojoj 
će ta virtualna mašina biti podignuta. Nadalje se sve može konfigurisati 
i preko komandne linije uz \textit{Kubernetes}-ov alat \texttt{kubectl}.
Preko ovog alata se izvršavaju komande sa pravljenje čaura.

Potrebno je napomenuti da se korišćenje \textit{Google Cloud Platform}-e,
naravno, naplaćuje. U trenutku pisanja ovog rada \textit{Google} za nove 
korisnike obezbeđuje besplatnih 300\$, koji su dovoljni za testiranje.
Svakako treba voditi računa prilikom podešavanja kako ne bi došlo do 
nepotrebnih troškova.


\chapter{Zaključak}\label{ch:zakljucak}

Cilj ovog rada je da predloži i obrazloži informacioni sistem u oblaku koji bi pomogao 
programerima i prevodiocima tokom razvijanja višejezične aplikacije. Korišćenjem ovog 
sistema smanjuje se jaz između ove dve grupe koje zajedničim snagama poboljšavaju 
korisničko iskustvo aplikacije na kojoj rade. Zbog sve prisutnije digitalizacije, 
ne sme se zaboraviti na grupe ljudi koje ne razumeju određeni jezik i zbog toga 
ne mogu da koriste aplikaciju.

Prilikom razvoja sistema korišćena je arhitektura mikroservisa. Ideja vodilja za odabir 
ove arhitekture je bila da informacioni sistem bude što je više moguće skalabilan, 
pouzdan i održiv. Sve veći trend korišćenja mikroservisa u oblaku za velike aplikacije 
potvrđuje da je to dobar izbor. 

Kako se računarstvo u oblaku razvija velikom brzinom, tako se razvijaju i novi alati 
koji pokušavaju da reše probleme koji nastaju na oblaku. S obzirom na sve veću upotrebu 
arhitekture mikroservisa, glavni problem predstavlja upravljanje većim brojem komponenti.
\textit{Kubernetes}, kao alat za rešavanje ovog problema, u potpunosti zadovoljava 
potrebe operacionog tima, ali i tima programera.

Alati za automatizaciju u mnogome pomažu razvoj softvera. Kontinuirano isporučivanje 
i kontinuirano raspoređivanje ubrzavaju ciklus razvoja i korisnici ranije mogu dobiti 
nove verzije softvera. Pored samih korisnika, benefit imaju i programeri jer mogu 
ranije da uoče probleme i brže da ih reše.

Potrebno je napomenuti da mikroservisi ne predstavljaju rešenje za sve. Iako se takvi 
sistemi lakše skaliraju i imaju bolju razdvojenost poslovne logike, sam razvoj ume da 
bude kompleksan. Prerano razmišljanje o optimizaciji i skaliranju sistema može usporiti 
razvoj i potencijalno dovesti do gašenja projekta jer funkcionalnosti sistema nisu 
isporučene na vreme. Sa druge strane, može se desiti da sam sistem uopšte nema potrebe 
da opsluži toliko korisnika. Onda se dobija kompleksan sistem bez prevelikog iskorišćenja,
što je potpuno suprotno od polazne pretpostavke. Potrebno je voditi se principom da se 
sistem optimizuje samo onda kada je to i zaista potrebno.


% ------------------------------------------------------------------------------
% Literatura
% ------------------------------------------------------------------------------
\literatura

% ==============================================================================
% Završni deo teze i prilozi
\backmatter
% ==============================================================================

% ------------------------------------------------------------------------------
% Biografija kandidata
\begin{biografija}
 \textbf{Uroš Milenković} (\emph{Bor,
   29. maj 1992.}) Završio prirodno--matematički smer, XIII beogradske gimnazije 2010. godine 
   i iste godine upisao Matematički fakultet u Beogradu. Osnovne studije, na smeru "Informatika",
   završava 2015, kada je upisao i master studije na istom smeru. Aprila 2016. počinje da radi 
   kao softverski inženjer u firmi \emph{CallidusCloud}, do septembra 2019. Tada prelazi u 
   firmu \emph{3ap}. Avgusta 2021. prelazi u firmu \emph{Nutanix}.
\end{biografija}
% ------------------------------------------------------------------------------

\end{document}

% !TeX root = ./matfmaster.tex
% Format teze zasnovan je na paketu memoir
% http://tug.ctan.org/macros/latex/contrib/memoir/memman.pdf ili
% http://texdoc.net/texmf-dist/doc/latex/memoir/memman.pdf
%
% Prilikom zadavanja klase memoir, navedenim opcijama se podešava
% veličina slova (12pt) i jednostrano štampanje (oneside).
% Ove parametre možete menjati samo ako pravite nezvanične verzije
% mastera za privatnu upotrebu (na primer, u b5 varijanti ima smisla
% smanjiti
\documentclass[12pt,oneside]{memoir}
\usepackage{minted}

% Paket koji definiše sve specifičnosti master rada Matematičkog fakulteta
\usepackage[latinica]{matfmaster}
%
% Podrazumevano pismo je ćirilica.
%   Ako koristite pdflatex, a ne xetex, sav latinički tekst na srpskom jeziku
%   treba biti okružen sa \lat{...} ili \begin{latinica}...\end{latinica}.
%
% Opicija [latinica]:
%   ako želite da pišete latiniciom, dodajte opciju "latinica" tj.
%   prethodni paket uključite pomoću: \usepackage[latinica]{matfmaster}.
%   Ako koristite pdflatex, a ne xetex, sav ćirilički tekst treba biti
%   okružen sa \cir{...} ili \begin{cirilica}...\end{cirilica}.
%
% Opcija [biblatex]:
%   ako želite da koristite reference na više jezika i umesto paketa
%   bibtex da koristite BibLaTeX/Biber, dodajte opciju "biblatex" tj.
%   prethodni paket uključite pomoću: \usepackage[biblatex]{matfmaster}
%
% Opcija [b5paper]:
%   ako želite da napravite verziju teze u manjem (b5) formatu, navedite
%   opciju "b5paper", tj. prethodni paket uključite pomoću:
%   \usepackage[b5paper]{matfmaster}. Tada ima smisla razmisliti o promeni
%   veličine slova (izmenom opcije 12pt na 11pt u \documentclass{memoir}).
%
% Naravno, opcije je moguće kombinovati.
% Npr. \usepackage[b5paper,biblatex]{matfmaster}

% Pomoćni paket koji generiše nasumičan tekst u kojem se javljaju sva slova
% azbuke (nema potrebe koristiti ovo u pravim disertacijama)
%\usepackage[latinica]{pangrami}

% Datoteka sa literaturom u BibTex tj. BibLaTeX/Biber formatu
\bib{matfmaster}

% Ime kandidata na srpskom jeziku (u odabranom pismu)
\autor{Uroš Milenković}
% Naslov teze na srpskom jeziku (u odabranom pismu)
\naslov{Razvoj veb aplikacije u oblaku za upravljanje prevodima u višejezičnim aplikacijama}
% Godina u kojoj je teza predana komisiji
\godina{2021}
% Ime i afilijacija mentora (u odabranom pismu)
\mentor{prof. dr Saša \textsc{Malkov}, vanredni profesor\\ Univerzitet u Beogradu, Matematički fakultet}
% Ime i afilijacija prvog člana komisije (u odabranom pismu)
\komisijaA{Anđelka \textsc{Zečević}, asistent\\ Univerzitet u Beogradu, Matematički fakultet}
% Ime i afilijacija drugog člana komisije (u odabranom pismu)
\komisijaB{doc. dr Jelena \textsc{Graovac}, docent\\ Univerzitet u Beogradu, Matematički fakultet}
% Ime i afilijacija trećeg člana komisije (opciono)
% \komisijaC{}
% Ime i afilijacija četvrtog člana komisije (opciono)
% \komisijaD{}
% Datum odbrane (odkomentarisati narednu liniju i upisati datum odbrane ako je poznat)
% \datumodbrane{}

% Apstrakt na srpskom jeziku (u odabranom pismu)
\apstr{%
Apstrakt
}

% Ključne reči na srpskom jeziku (u odabranom pismu)
\kljucnereci{računarstvo u oblaku, višejezičnost}

\begin{document}
% ==============================================================================
% Uvodni deo teze
\frontmatter
% ==============================================================================
% Naslovna strana
\naslovna
% Strana sa podacima o mentoru i članovima komisije
\komisija
% Strana sa posvetom (u odabranom pismu)
\posveta{Mami, tati i dedi}
% Strana sa podacima o disertaciji na srpskom jeziku
\apstrakt
% Sadržaj teze
\tableofcontents*

% ==============================================================================
% Glavni deo teze
\mainmatter
% ==============================================================================

\chapter{Uvod}\label{ch:uvod}

Aplikacije koje su namenjene za globalno tržište treba pripremiti 
tako da korisnici koji pripadaju različitim govornim područijima i 
geografskim regionima razumeju sadržaj. Postupak dizajniranja aplikacije 
tako da podrži različite jezike se naziva internacionalizacija, a osobina 
da aplikacije koja podržava više jezika se naziva višejezičnost. 

Proces prilagođavanja softvera može biti kompleksan. Veliku pomoć u tom procesu
danas pružaju različita rešenja. Odabir rešenja uglavnom zavisi od odabira 
programskog jezika u kom će softver biti razvijan, kao i od korišćenja 
programskog okvira.

Sa druge strane, ne može se očekivati da programer zna sve jezike za koje je 
softver namenjen. To znači da se u proces internacionalizacije uključuju i 
prevodioci koji poznaju jezik na koji softver treba prevesti. Oni implicitno 
postaju osobe koje razvijaju softver.

Pomenuta rešenja za olakšavanje internacionalizacije uglavnom su fokusirana
samo na tehničke probleme, odnosno probleme programera. Prevodi se čuvaju u
fajlovima i obično predstavljaju serijalizovanu strukturu mape, odnosno parove
"ključ -- vrednost". Takvi fajlovi su neretko teški za korišćenje od strane
netehnikičih lica, a prevodioci često nisu tehnička lica. Kako bi što bolje 
obavljali svoj posao, prevodiocima je potrebna neka vrsta alata koja im deluje
poznato, nešto na šta su navikli, odnosno nešto što koriste svaki dan. 
Prevodioce ne zanima koje rešenje su programeri izabrali, niti koji format 
serijalizacije se koristi. Njima je potreban familijaran interfejs za ažuriranje
tih fajlova.

Razvojem računarstva u oblaku, ljudi su navikli da im se sve nalazi u oblaku, 
odnosno da im je sve uvek dostupno, na svakom mestu i u bilo koje vreme. 
Paralelno sa rezvojem softvera u oblaku, popularnost stiču arhitekture zasnovane
na mikroservisima ali i razvoj aplikacija u kontejnerima. Za kontrolisanje 
velikog broja kontejnera se sve više korsti alat Kubernetes. Pored toga, 
tendencija je da se hardver, ondnosno infrastruktura, više ne održava u okviru
organizacije, već da se on iznajmljuje od dobavljača računarstva u oblaku 
(eng. \textit{cloud computing provider}). Ovakvim pristupom se omogućava lako 
prenošenje aplikacije sa jednog okruženja na drugo. Razvoj takvog softvera nosi
sa sobom dodatne izazove. 

U ovom radu biće opisan razvoj softvera u oblaku. Fokus će biti stavljen na 
arhitekturu mikroservisa, korišćenje alata Kubernetes i proces pravljenja 
višejezične aplikacije. Za razvoj klijentskog, ali i serverskog dela 
kôda, koristiće se programski jezik JavaScript. U svrhu ilustracije i
boljeg razumevanja biće razvijena aplikacija za upravljanje prevodima. 
Aplikacija je nazvana Interpres i biće razvijana kao softver otvorenog kôda.

\chapter{Analiza}\label{ch:analiza}

Analiza

\chapter{Funkcionalni zahtevi}\label{ch:funkcionalni_zahtevi}

Funkcionalni zahtevi

\chapter{Arhitektura i dizajn sistema}\label{ch:arhitektura}

Funkcionalni zahtev da aplikacija bude u oblaku, opisan u poglavlju 
\ref{sec:funkcionalni_zahtevi-aplikacija_u_oblaku}, nameće grubu sliku kako arhitektura sistema 
treba da izgleda. Aplikacije na vebu su dominantno bazirane po modelu klijent--server arhitekture. 
Server u takvoj arhitekturi obezbeđuje uslugu klijentima koji je zahtevaju. Klijent--server 
arhitektura je višenamenska i modularna, a sa ciljem unapređenja upotrebljivosti, interoperabilnosti, 
fleksibilnosti i skalabilnosti. 

Veb aplikacije su nezavisne od platforme jer zahtevaju samo veb pregledač, dok se kod standardne 
klijent--server arhitekture klijent mora instalirati na platformi korisnika. Pored toga, 
kod veb aplikacija je protokol komunikacije definisan, koristi se \textit{HTTP}, dok se kod klijent--server
arhitekture protokol može izabrati po potrebi.

Komponente klijent -- server arhitekture, prilagođene za veb, se mogu grupisati u tri kategorije: 
server, klijent i mreža. Uloga servera je da upravlja zajedničkim resursima, bazom podataka, da 
izvršava poslovnu logiku, kao i da kontroliše pristup i bezbednost podataka. Posao klijenta je da 
upravlja korisničkim interfejsom. Računarska mreža omogućava komunikaciju između klijenta i servera, 
a komunikacija mora pratiti određene standarde.

\section{Arhitektura teških klijenata}\label{sec:arhitektura-spa}

Odlika arhitekture teških klijenata je da server šalje klijentu podatke i meta podatke, i prepušta 
mu da samostalno pripremi prezentaciju.~\cite{PVEB} Ovakvom podelom odgovornosti se smanjuje opterećenje 
servera, jer više ne mora da priprema prezentaciju. Arhitektura teških klijenata podrazumeva intenzivno 
korišćenje \textit{JavaScript}-a.

Ekstremni oblik ove arhitekture je arhitektura jedne stranice (eng. \textit{SPA - Single Page Application}).
Arhitektura jedne stranice propisuje da se čitava aplikacija učitava kroz jednu stranicu. Primenom 
ovih pravila se dobija potpuno odvojen klijent od servera. Klijent u ovom slučaju komunicira 
sa serverom u pozadini radi preuzimanja podataka i izvođenja transakcija. S obzirom da prezentacija
priprema na korisničkoj strani, korisnički interfejs ima bolji odziv, jer se ne prenosi kroz mrežu.

\section{Mikroservisi}\label{sec:arhitektura-mikroservisi}

Tradicionalan način razvijanja poslovnih aplikacija predstavlja razvoj takozvane monolitne arhitekture. 
Jedan monolit bi sadržao svu poslovnu logiku koju izvršava aplikacija. Kako aplikacija raste vremenom, 
tako raste i sam monolit, koji se sve teže održava. Problemi koji nastaju kod velikog monolita nisu 
samo problemi održavanja, već je i skaliranje upitno. Njih karakteriše spor razvojni ciklus i ažuriraju se relativno retko.

Danas se ovako veliki monoliti raščlanjuju na manje, nezavisne komponente koje se nazivaju mikroservisima. 
Džejms Luis i Martin Fauler kažu da se izraz "Mikroservisna arhitektura" sve češće pojavlje poslednjih 
par godina kako bi opisao određeni način projektovanja softverskih aplikacija kao paketa usluga koji se 
mogu odvojeno isporučiti. Iako ne postoji precizna definicija ovog arhitektonskog stila, postoje određene 
zajedničke karakteristike u organizaciji oko poslovne logike, automatskog isporučivanja i decentralizovane 
kontrole podataka.~\cite{martinfowler_microservices} 

Mikroservisi predstavljaju aplikaciju koja je struktuirana kao kolekcija slabo vezanih servisa. Glavna 
ideja iza mikroservisne arhitekture je da se aplikacije lakše razvijaju i održavaju ako su podeljene 
na manje delove koji rade nesmetano zajedno.

Servisi se mogu posmatrati kao komponente sistema. Martin Fauler definiše komponente kao jedinice 
softvera koje se mogu nezavisno zameniti i unaprediti.~\cite{martinfowler_software_component}.

Biblioteke su komponente koje su povezane u program i pozivaju se pomoću funkcija koje se nalaze u 
memoriji, dok su servisi komponente koje su van procesa i koje komuniciraju preko mreže.
Servisi, naravno, mogu da koriste biblioteke. Glavni benefit ovakvog korišćenja servisa je da se mogu 
nezavisno isporučiti, što nije slučaj ako imamo jednu aplikaciju koja je sastavljena od skupa biblioteka.
Sa druge strane, daljinski pozivi preko mreže su skuplji, odnosno sporiji. 
Druga loša strana je da se logika teže prebacuje iz jednog servisa u drugi, odnosno refaktorisanje je 
otežano u tom smislu.

Da bi se monoliti lakše razvijali, programeri se uglavnom podele po tehnologiji, na primer u tim za 
bazu podataka, tim za poslovnu logiku i tim za korisnički interfejs. U arhitekturi mikroservisa je malo 
drugačije. Timovi se organizuju oko jednog servisa i sastavljeni su od progamera različitih tehnologija.
To znači da su timovi nezavisni, kao i što su i sami servisi nezavisni, i organizovani su sa fokusom na 
poslovnu logiku, a ne na tehnologiju.

Jedna zgodna posledica deljenja monolita na servise je da servisi ne moraju da budu razvijeni u istim 
tehnologijama. Tako na primer, jedan servis može biti izgrađen u programskom jeziku \textit{C++} a drugi u 
\textit{NodeJS}, dokle god mogu da komuniciraju jedan sa drugim. Nekada je pogodnije koristiti drugi programski 
jezik jer je u njemu lakše rešiti problem.

Monolitne aplikacije preferiraju da imaju jednu bazu podataka za čuvanje podataka. Sa druge strane, kod 
mikroservisnih aplikacija se odluka o bazi podataka prepušta samom servisu. Neke probleme koje servis 
rešava je pogodnije rešiti relacionom bazom podataka, dok je za neki drugi servis možda pogodnija grafovska baza.
Iako i monolitne aplikacije mogu da koriste više tipova baza podataka, ova osobina je češća kod 
mikroservisa. Decentralizovana odgovornost za čuvanje podataka ima i svoje implikacije na ažuriranje 
podataka. Ovaj problem se može rešiti korišćenjem transakcija. Korišćenjem transakcija rešava se problem 
konzistentnosti, ali se smanjuje dostupnost. Zato, česta je odluka da se konzistentnost zameni 
odloženom konzistentnošću, u onim slučajevima gde je ona moguća. 

Kada se govori o mikroservisima često se, sa razlogom, postavlja pitanje postoji li razlika između arhitekture
mikroservisa i servisno orijentisane arhitekture (eng. \textit{Service Oriented Architecture}, \textit{SOA}). 
Glavne karakteristike arhitekture mikroservisa su umnogome slične \textit{SOA}. Problem je taj što \textit{SOA} 
može da predstavlja mnogo različitih stvari. Preveliki fokus na kanal komunikacije 
(eng. \textit{Enterprise Service Bus}, \textit{ESB}) koji se koristi za integraciju monolitnih aplikacija, 
predstavlja jedan od tih problema. Može se reći da je arhitektura mikroservisa nastala iz stečenog iskustva tokom 
razvijanja \textit{SOA}, odnosno da predstavlja sledeći iterativni korak u razvoju ove arhitekture. 
Preuzeti su raznorazni dobri obrasci iz \textit{SOA}. Sa druge strane, delovi koji su bili previše kompleksni 
zamenjeni su jednostavnijim. Primer je \textit{ESB} koji je zamenjen jednostavnijim veb protokolima. 

Ljudi koji zagovaraju arhitekturu mikroservisa su iz navedenih razloga krenuli da odbacuju naziv \textit{SOA}, 
dok drugi smatraju da su mikroservisi samo još jedan oblik \textit{SOA}. Baš zato što \textit{SOA} 
može da predstavlja toliko različitih stvari, dragoceno je imati izraz koji preciznije opisuje arhitekturu 
zvanu mikroservisi.~\cite{martinfowler_microservices}

\section{REST}\label{sec:arhitektura-rest}

Interoperabilnost je sposobnost da različiti sistemi rade zajedno. Da bi se ovo postiglo potrebni su 
standardi koje će sistemi poštovati. Kako u klijent -- server arhitekturi imamo dva različita sistema 
(klijent i server), oni moraju poštovati određene standarde za komunikaciju. Jedan od takvih standarda za 
komunikaciju je i \textit{REST} (eng. \textit{Representational State Transfer}). \textit{REST} je stil 
softverske arhitekture koji definiše skup pravila koja treba da se poštuju tokom pravljenja mrežnih 
aplikacija. Prvi put je predstavljen u doktorskoj disertaciji Roja Fildinga~\cite{REST_Roy}. Po \textit{REST}-u, 
fokus se prebacuje sa procedura na resurse (objekte). Za aplikaciju kažemo da je \textit{RESTful} ako poštuje pravila REST-a.


\textit{REST} definiše šest pravila koja moraju da se ispoštuju~\cite{REST_API}:

\begin{itemize}

	\item \textbf{Klijent -- server}: Razdvajanjem korisničkog interfejsa od servera briga o podacima 
	ostaje na serveru. Klijenti postaju lakši za portabilnost među različitim sistemima, a server se može
    lakše skalirati.
	
    \item \textbf{Sistem bez stanja (eng. \textit{Stateless})}: Svaki zahtev sa klijenta ka serveru mora da sadrži 
	sve potrebne informacije da bi zahtev bio opslužen. Klijent se ne sme oslanjati na stanje servera. 
    Zapravo, stanje na serveru se zabranjuje. Iz tog razloga stanje sesije se u potpunosti čuva na klijentu.
	
    \item \textbf{Keširanje}: Server u svom odgovoru sa nekim podacima mora eksplicitno da navede da li 
    podaci mogu da se keširaju, ili ne. Ako server odgovori da su podaci kešabilni, klijent ih može 
    koristiti kasnije ako je zahtev za podacima isti. Server može napomenuti i vreme isteka keša.
    
    \item \textbf{Uniforman interfejs}: Uniforman interfejs definiše interfejs između klijenta i servera.
    On pojednostavljuje i razdvaja arhitekturu, što omogućava da se svaki deo razvija samostalno. 
    Četiri vodeća principa uniformnog interfejsa su:

    \begin{enumerate}
        \item \textbf{Zasnovanost na resursima}: Individualni resursi se mogu identifikovati u 
        zahtevima koristeći \textit{URI} kao identifikator resursa. Resursi kao takvi su konceptualno odvojeni 
        od reprezentacije koja se vraća klijentima. Na primer, server ne šalje svoju bazu podataka, 
        već radije, šalje \textit{HTML}, \textit{XML} ili \textit{JSON} koji predstavljaju traženi zapis iz baze podataka.

        \item \textbf{Upravljanje resursima kroz reprezentaciju}: Kada klijent sadrži reprezentaciju resursa,
        uključujući meta podatke u prilogu, onda ima dovoljno podataka da izmeni ili obriše resurse 
        sa servera, pod uslovom da ima ovlašćenje da to uradi.

        \item \textbf{Samoopisne poruke}: Svaka poruka sadrži dovoljno informacija da opiše na koji 
        način treba da bude obrađena. Na primer \textit{Internet media type} (nekada poznat pod nazivom \textit{MIME}) 
        može sadržati informaciju o tome koji parser treba da se pozove. Odgovori eksplicitno 
        označavaju da li imaju sposobnost keširanja.

        \item \textbf{Hipermedija kao pokretač aplikacije (\textit{HATEOAS})}:
        Klijenti treba da znaju što je manje moguće o tome kako da komuniciraju sa serverom. Komunikacija treba da bude
        što je više moguće generička. Server svoje stanje isporučuje klijentu u vidu hiperteksta, koji unutar sebe 
        sadrži hiperlinkove. To se tehnički naziva hipermedija (hiperlinkovi unutar hiperteksta). 
        \textit{HATEOAS} (eng. Hypermedia as the Engine of Application State) 
        znači da, tamo gde je potrebno, veze su sadržane unutar vraćenog odgovora. 
        Tako se isporučuje URI za preuzimanje samog objekta ili srodnih 
        objekata. Uniformni interfejs koji svaki \textit{REST} servis mora pružiti je fundamentalna 
        osnova za njegov dizajn.
        
        
    \end{enumerate}
    
	\item \textbf{Slojevit sistem}: \textit{REST} omogućava korišćenje slojevitih sistema, gde klijent ne zna 
    eksplicitno da li komunicira sa krajnjim serverom ili sa posrednikom. Ovako se može povećati 
    skalabilnost uvođenjem balansera opterećenja ili uvođenjem keširanja na strani servera. 
	
	\item \textbf{Kod na zahtev (opciono)}: Serveri mogu obogatiti klijentsku stranu slanjem koda koji će biti izvršen na 
	klijentskoj strani. Ovo može uprostiti klijente jer se smanjuje broj funkcija koje je potrebno da 
    klijent ima. Primer na vebu bi bio slanje \textit{JavaScript} koda. 

\end{itemize}



\chapter{Tehnologije}\label{ch:tehnologije}

U ovom poglavlju će biti opisane tehnologije korišćene za razvoj rešenja.

\section{React}\label{sec:react}

\section{Material UI}\label{sec:material_ui}

\section{NestJS}\label{sec:nestjs}

\section{GitHub aplikacija}\label{sec:github_app}

\section{Docker}\label{sec:docker}

Pre korišćenja kontejnera, glavni naćin za izolovanje, organizaciju aplikacije i njenih zavisnosti je 
bio postavljanje svake aplikacije na zasebnu virtuelnu mašinu. Takve mašine su pokretale više aplikacija 
na istom fizičkom hardveru i takav proces se naziva virtuelizacija.

Virtuelizacija je imala nekoliko nedostataka: 
\begin{itemize}
    \item Virtuelne mašine su bile glomazne
    \item Pokretanjem više virtuelnih mašina uticalo je na performanse
    \item Sam proces pokretanja je predugo trajao
\end{itemize}

Pomenuti nedostaci doveli su do nastanka nove tehnike korišćenja kontejnera (kontejnerizacije). 

Kontejnerizacija je tip virtuelizacije koja dovodi virtuelizaciju na nivo operativnog 
sistema. Kao sto virtuelizacija abstrahuje hardver, tako i kontejnerizacija abstrahuje 
operativni sistem.

Neke od prednosti kontejnerizacije su sledeće:
\begin{itemize}
    \item Kontejneri nemaju gostujući operativni sistem i koriste operativni sistem domaćina. 
    Dele relevantne biblioteke i resurse onda kada je to potrebno.
    \item Procesiranje i izvršavanje aplikacije je veoma brzo jer se komplajlirana aplikacija 
    i biblioteke kontejnera izvršavaju na kernelu domaćina.
    \item Pokretanje kontejnera traje samo delić sekunde. Takođe, kontejneri su lakši i brži 
    od virtuelnih mašina.
\end{itemize}

Docker je platforma koja aplikaciju i sve njene zavisnosti pakuje u formi kontejnera. 
Ovakav aspekt obezbeđuje da aplikacija radi na svim okruženjima.

Svaka aplikacija se pokreće na zasebnom kontejneru i ima svoj skup zavisnosti i biblioteka. 
Zbog toga možemo biti sigurni da svaka aplikacija radi nezavisno od ostalih aplikacija, 
dajući programerima sigurnost da grade aplikacije koje neće posredovati međusobno.

Docker fajl je tekstualni dokument koji sadrži sve komande koje korisnik može pokrenuti 
u komandnoj liniji kako bi se sastavila Docker slika. Docker može izgraditi sliku automatski 
na osnovu pročitanih instrukcija iz Docker fajla.

Docker sliku možemo uporediti sa šablonom koji se koristi za kreiranje Docker kontejnera. 
To su šabloni koji se ne mogu menjati i predstavljaju gradivni element kontejnera.
Docker slike se čuvaju u Docker registrima. 

Docker kontejner je pokrenuta instanca Docker slike. Sadrži sve što je potrebno da bi se 
pokrenula aplikacija. To je u osnovi spremna aplikacija kreirana iz Docker slike, 
što ujedno predstavlja i krajnji produkt Docker-a.~\cite{docker}

Docker je trenutno najpopularnija implementacija kontejnera.

\section{Kubernetes}\label{sec:kubernetes}

Do nedavno, većina softverskih aplikacija je razvijana kao veliki monoliti, koji su funkcionisali kao jedan proces 
ili kao mali broj procesa rasprostanjenih na više servera. Ovi zastareli sistemi su i dalje veoma rasprostranjeni. 
Njih karakteriše spor razvojni ciklus i ažuriraju se relativno retko. Na kraju svakog razvojnog ciklusa, programeri
upakuju ceo sistem i predaju ga timu zaduženom za operacije, koji ga kasnije instalira i nadgleda. U slučaju hardverske
greške, tim za operacije ručno migrira sistem na preostale servere koji su bez greške.

Danas se ovako veliki monoliti rasčlanjuju na manje, nezavisne komponente koje se nazivaju mikroservisima. S obzirom
da su mikroservisi odvojeni jedni od drugih, mogu se razvijati, instalirati, ažurirati i skalirati svaki ponaosob.
Ovakva osobina omogućava češće promene na komponentama. S druge strane, povećanjem broja komponenti koje treba 
instalirati postaje sve teže konfigurisati, upravljati i očuvati ceo sistem u radnom stanju. Pored navedenog, mnogo 
je teže shvatiti kako i gde postaviti ove komponente kako bi se postigla veća iskorišćenost resursa, a samim tim i 
smanjiti cenu potrebnog hardvera. Odatle postoji potreba za automatizacijom, koja uključuje automatsku konfiguraciju,
nadzor  i rešavanje problema. Iz ovih razloga je razvijen Kubernetes.

Kubernetes omogućava programerima da sami instaliraju svoju aplikaciju, bez pomoći tima za operacije. Ali s druge 
strane, nemaju samo programeri benefit. Ovaj alat takođe pomaže operacionom timu tako što automatski nadgleda, 
i u slučaju greške pokreće nove instance aplikacija. To znači da se fokus operacionog tima preusmerava sa nadgledanja
pojedinačnih aplikacija na nadgledanje i upravljanje infrastrukture i Kubernetes alata, dok se Kubernetes stara o samim
aplikacijama.

Kubernetes apstrahuje hardversku infrastrukturu i pruža privid da je ceo data centar jedan veliki resurs. To omogućava
velikim kompanijama koje pružaju usluge računarstva u oblaku da ponude programerima jednostavnu platformu za pokretanje
raznih tipova aplikacija, a da pritom njihovi administratori sistema ne znaju koje su aplikacije pokrenute na njihovom
hardveru. Kako velike kompanije sve više prihvataju Kubernetes model kao jedan od boljih načina za pokretanje aplikacija,
tako Kubernetes postaje standardan model za računarstvo u oblaku~\cite{KIA}.

Kubernetes je softver otvorenog koda koji služi za okestraciju kontejnera, a razvijen je od 
strane Google-a. Pomaže pri upravljanju aplikacijama koje su razvijene u velikom broju 
kontejnera. Može da se primeni na različita okruženja za isporuku, kao što su fizički 
hardver, virtuelne mašine ili oblak.

Razvojem mikroservisnih arhitektura dovelo je do povećane upotrebe kontejner tehnologija, jer 
kontejneri predstavljaju savršeno rešenje za male, nezavisne aplikacije, kao što su mikroservisi. 
To je dalje dovelo do toga da se aplikacije sada nalaze u velikom broju kontejnera. Upravljanje 
tim kontejnerima, kroz različita oruženja, uz pomoć skripti ili alata koji su nastali u okviru 
kompanije koja proizvodi aplikaciju postaje ubrzo jako kompleksno.

Prednosti korišćenja Kubernetes su mnogobrojne. Visoka dostupnost aplikacije je jedna od tih prednosti. 
To zači da će korisnici moći (skoro) uvek da pristupe aplikaciji. Druga prednost je horizontalna 
skalabilnost aplikacije, odnosno po potrebi se lako dodaju novi čvorovi sistemu. Treća prednost koja 
dolazi uz korišćenje Kubernetes je oporavak od otkazivanja, što praktično znači da, ako je došlo do 
greške u infrastrukturi, Kubernetes ima mehanizme da bekapuje podatke i da nastavi sa radom od 
poslednjeg sačuvanog stanja.

\subsection{Arhitektura Kubernetes-a}
Arhitektura Kubernetes-a počinje od klastera. Klaster predstavlja skup čvorova. Svaki klaster sadrži
jedan glavni čvor koji je povezan sa jednim ili više radnih čvora. Radni čvorovi na sebi imaju 
takozvani "kublet" porces koji je pokrenut na njima. Ovaj proces služi da klaster moze da komunicira
sa radnim čvorovima i izvšava određene poslove na njima, kao što je pokretanje procesa za aplikacije.
Svaki radni čvor ima različite Docker kontejnere, različitih aplikacija, koje su isporučene na njemu.
Raspored kontejnera u radnim čvorovima zavisi od opterećenja sistema. Ako je opterećenje za određeni 
servis veće, Kubernetes može da pokrene veći broj kontejnera za taj servis. 

Dok se aplikacija izvršava na radnim čvorovima, glavni čvor služi da se na njemu pokrenu bitni procesi 
bez kojih Kubernetes ne može da radi. Jedan od tih procasa je {\em API Server}, koji služi za 
komunikaciju sa različitim Kubernetes klijentima, kao što je korisnički interfejs ili alat u 
komandnoj liniji. Drugi proces koji se nalazi na glavnom čvoru je {\em Controller manager} koji 
prati šta se dešava u klasteru,da li nešto treba da se popravi, ili da otkrije da je došlo do greške 
u kontejneru. Dalje, na glavnom čvoru se nalazi proces pod imenom {\em Scheduler} koji je zadužen 
za podizanje kontejnera na različitim čvorovima u odnosu na opterećenje i dostupne resurse na svakom 
čvoru. Scheduler je pametan proces koji odlučuje na kom čvoru će biti podignut koji kontejner. 
Još jedna bitna komponenta na glavnom čvoru je {\em etcd}, "ključ -- vrednost" skladište, koje služi 
da čuva stanje Kubernetes klastera. On čuva sve konfiguracije za svaki čvor, aplikaciju, ali i statusne 
podatke o svakom kontejneru. Poslednja, ali nimalo manje važna komponenta u arhitekturi Kubernetes-a, 
je virtuelna mreža. Preko nje čvorovi mogu međusobno da komuniciraju. 

Može se primetiti da će glavni resursi biti raspoređeni na radne čvorove, jer oni služe za pokretanje 
aplikacije. Za glavni čvor nije potrebno toliko resrursa, jer se na njemu pokrecu ne toliko zahtevni procesi 
koji služe samo za rad Kubernetesa. Ako nekim slučajem dođe do otkazivanja radnog čvora, glavni čvor 
će se pobrinuti da se podigne novi radni čvor sa istom konfiguracijom. S druge strane, ako dođe do 
otkazivanja glavnog čvora, gubimo konekciju sa svim ostalim čvorovima. Iz tog razloga, u produkcionom 
okruženju se uvek drži barem 2 pokrenuta glavna čvora, tako da, ako jedan padne, drugi preuzima 
posao na sebe.

\subsection{Osnovni koncepti Kubernetes-a}
{\em Pod} u Kubernetes-u predstavnja najmanju jedinicu koja može da se konfiguriše i sa kojom može da 
se ostvari interakcija. On praktično predstavlja omotač oko kontejnera. U okviru jednog radnog čvora 
može se naći više Pod-ova, a u jednom Pod-u se može naći više kontejnera. Uobičajeno je da se jedan 
kontejner nalazi u jednom Pod-u, ali postoje slučajevi kada jedan kontejner zahteva pomoćne kontejnere 
i tada se može naći više njih u jednom Pod-u. To zači da će jedna aplikacija, odnosno jedan servis, 
biti u jednom Pod-u. Pod predstavlja abstrakciju za upravljanje nad kontejnerom koji se poreće unutar
Pod-a. Na primer, ako se kontejner ugasi, Pod će ga ponovo podići za nas. 

Pod-ovi predstavljaju privremene komponente, što znači da često mogu da otkažu iz različitih razloga. 
Na primer, kada je potrebno isporučiti novu verziju aplikacije, prvo će se kreirati novi Pod-ovi sa 
novom verzijom, a potom će se stari ukloniti. Pomenuta virtuelna mreža koja se nalazi nad celim 
klasterom će svakom Pod-u dodeliti po jednu IP adresu. To znači da je svaki Pod zaseban server sa 
svojom IP adresom preko koje međusobno komuniciraju.

S obzirom da se Pod-ovi predstavljaju privremene komponente, i da se kreiranjem novog Pod-a dodeljuje 
nova IP adresa, prirodno se uvodi pojam {\em Servisa}. Servis predstavlja zamenu za IP adrese, tako 
da umesto da podovi komuniciraju međusobno preko IP adrese, oni mogu da komuniciraju preko Servisa 
koji dalje prosleđuju komunikaciju Pod-u. Tako da, ako se Pod rekreira, ostali će znati da komuniciraju 
s njim kada ponovo bude dostupan. Pored zamene IP adrese, Servisi služe i kao balanseri opterećenja. 

\subsection{Konfiguracija Kubernetes-a}
Konfigurisanje Kubernetes sistema se odvija deklarativno, uz pomoć YAML fajla. U njemu deklarišemo
koji kontejner treba da se podgne, od koje Docker slike treba da se napravi, koliko Pod-ova treba 
da bude u svakom trenutku, kao i promenljive iz okruženja. Kubernetes se stara da ovi zahtevi budu 
ispoštovani, i za to je zadužen Controller Manager, koji proverava konfiguraciju i upravlja ostalim 
procesima. 

\section{Kontinualna integracija, isporuka i raspoređivanje}\label{sec:arhitektura-ci_cd}

CI/CD je metod za često dostavljanje aplikacija korisnicima kroz predstavljanje automatizacije u 
fazama razvoja aplikacije. Glavni koncept CI/CD su kontinualna integracija (eng. 
\textit{continuous integration}), kontinualno dostavljanje (eng. \textit{continuous delivery}) i 
kontinualno raspoređivanje (eng. \textit{continuous deployment}). CI/CD je rešenje za problem 
koji nastaje prilikom integracije novog koda.

CI/CD uvodi automatizaciju i kontinualno praćenje kroz životni ciklus aplikacije, 
od integracije i testne faze do dostavljanja i raspoređivanja. Zajedno, ove povezane prakse 
često se nazivaju “CI/CD tok” (eng. \textit{CI/CD pipeline}), a podržani su i od strane razvojnih 
i od strane operativnih timova.

\subsection{Razlika između CI i CD}
Akronim CI/CD ima više različitih značenja. “CI” u CI/CD uvek označava kontinualnu integraciju, 
koja predstavlja automatizovan proces za programere. Uspešan CI označava da su nove izmene koda 
na aplikaciji regularno izgrađene, testirane i pripojene deljenom repozitorijumu. To je rešenje 
za problem koji postoji onda kada ima previše grana u razvoju koje mogu dovesti do konflikata.

“CD” u CI/CD označava kontinualno dostavljanje i/ili kontinualno raspoređivanje, koji su povezani 
koncepti i mogu se koristiti naizmenično. I jedan i drugi tiču se automatizacije faza toka, međutim, 
nekada se koriste i odvojeno u cilju prikaza količine automatizacije koja se odvija.

Kontinualno dostavljanje obično znači da su izmene koje je programer napravio automatski testirane 
i otpremljene na repozitorijum, gde se onda mogu rasporediti na produkciono okruženje od strane 
operacionog tima. Može se smatrati odgovorom na slabu preglednost i komunikaciju između tima 
programera i poslovnog tima. Iz tog razloga, svrha kontinualnog dostavljanja jeste da omogući 
minimalne napore za raspoređivanje novog koda.

Kontinualno raspoređivanje odnosi se na automatsko puštanje izmena sa repozitorijuma na produkciju. 
Rešava problem preopterećenja operacionog tima manuelnim procesima koji usporavaju dostavljanje 
aplikacije. Zasniva se na prednostima kontinualnog dostavljanja automatizacijom narednih faza u toku.

\begin{figure}[h]
    \centering
    \includegraphics[width=0.75\textwidth]{ci-cd-flow-desktop}
    \caption{CI/CD tok}
\end{figure}

Moguće je da CI/CD obuhvati povezane prakse kontinualne integracije i kontinualnog dostavljanja, 
ili sve 3 povezane prakse kontinualne integracije, kontinualnog dostavljanja i kontinualnog 
raspoređivanja. Dodatnu komplikaciju stvara to što se kontinualna isporuka može nekada koristiti 
na način koji obuhvata procese kontinualnog raspoređivanja.

CI/CD je u stvari proces, često predstavljen kao tok, koji obuhvata uvođenje visokog stepena 
automatizacije i kontinualnog praćenja razvoja aplikacije.

Od slučaja do slučaja, na šta se termini konkretno odnose zavisi od toga koliko je automatizacije 
ugrađenou CI/CD tok. Mnoga preduzeća započinju dodavanjem CI, nakon čega uvode automatizaciju 
dostavljanja i raspoređivanja.

\subsection{Kontinualna integracija}
U razvoju modernih aplikacija, cilj je da više programera istovremeno radi na različitim delovima 
aplikacije. Međutim, ukoliko je organizacija postavljena tako da se spajanje koda sa svih grana 
vrši u jednom danu, takav posao može biti monoton, manuelan i dugotrajan. To se dešava u slučajevima 
kada programer vrši izmene na aplikaciji i na taj način povećava šansu za nastajanje konflikta sa 
izmenama koje istovremeno prave drugi programeri.

Kontinualna integracija (CI) pomaže programerima da spoje izmene na kodu na deljenu granu češće, 
čak i na dnevnom nivou. Nakon spajanja izmenjenih delova koda, izmene se validiraju tako što se 
automatski gradi aplikacija i pokreće se više nivoa automatskog testiranja. To znači da se testira 
sve, od klasa i funkcija do različitih modula koji su deo aplikacije. Ako automatski testovi pronađu 
konflikt između novog i postojećeg koda, CI nudi lako rešavanje konflikata.

\subsection{Kontinualna dostavljanje}
Nakon automatskog građenja aplikacije i testiranja u CI, kontinualno dostavljanje automatizuje 
puštanje prethodno validiranog koda na repozitorijum. Kako bismo imali efektivan proces kontinualne 
dostave, važno je da imamo već ugrađen CI u protoku. Cilj kontinualnog dostavljanja jeste postojanje 
baze koda koja je uvek spremna za raspoređivanje na produkciono okruženje.

U kontinualnom dostavljanju, svaka faza, počevši od spajanja izmenjenog koda do dostavljanja 
verzija spremnih za produkciju, podrazumeva automatsko testiranje i automatizaciju raspoređivanja 
koda. Na kraju tog procesa, operacioni tim je u mogućnosti da brzo i jednostavno rasporedi 
aplikaciju na produkciju.

\subsection{Kontinualno raspoređivanje}
Poslednja faza CI/CD protoka je kontinualno raspoređivanje. Kao dodatak kontinualnom dostavljanju, 
koji automatizuje isporuku verzija spremnih za produkciju, kontinualno raspoređivanje automatizuje 
puštanje aplikacije u produkciju. U velikoj meri se oslanja na dobro osmišljeno automatsko testiranje.

U praksi, kontinualno raspoređivanje znači da izmene na aplikaciji mogu biti na produkciji za samo 
nekoliko minuta (pod pretpostavkom da su automatski testovi uspešno završeni).

Sve povezane prakse CI/CD čine raspoređivanje aplikacije manje rizičnim, stoga je lakše pustiti 
promene u aplikaciji u delovima, pre nego odjednom.~\cite{CI_CD}

\subsection{GitHub Akcije}
Za verzionisanje koda se može koristiti GitHub. Pored verzionisanja, on pruža i alate za automatizaciju 
kontinualne integracije, isporue i raspoređivanja, kroz alat koji su nazvali \textit{Github Akcije} 
(eng. \textit{Github Actions}). 
GitHub Akcije su zasnovane na događajima, što znači da mogu da pokrenu niz komandi koje će se desiti 
posle navedenog događaja. Na primer, svaki put kada neko napravi \textit{zahtev za promenu} 
(eng. \textit{Pull Request}) nad repozitorijumu, može se automatski pokrenuti komanda koja pokreće 
testove. 

GitHub Akcije koriste YAML fajlove kako bi se definisale komponente toka rada. Ovi fajlovi 
se čuvaju u repozitorijumu, u folderu pod nazivom \mbox{\texttt{.github/workflows}}.

Komponente koje postoje u GitHub Akcijama su: 

\begin{itemize}
    \item Tok rada (eng. \textit{Workflow})
    \item Događaj (eng. \textit{Event})
    \item Posao (eng. \textit{Job})
    \item Korak (eng. \textit{Step})
    \item Akcija (eng. \textit{Action})
    \item Izvršilac (eng. \textit{Runner})
\end{itemize}

\begin{figure}[h]
    \centering
    \includegraphics[width=0.5\textwidth]{overview-actions-design}
    \caption{Prikaz interakcija komponenata GitHub Akcija}
\end{figure}

\subsubsection{Tok rada}
Tok rada je automatizovana procedura koja se dodaje na repozitorijum. Tokovi radsa su sastavljeni 
od jednog ili više poslova i mogu se pokrenuti na određeni događaj ili biti zakazani u određeno vreme. 
Tok rada se može koristiti da se aplikacija izgradi, testira, spakuje, isporuči ili rasporedi na 
različita okruženja.

\subsubsection{Događaji}
Događaj je specifična aktivnost koja pokreće tok rada. Na primer, aktivnost može nastati kad neko napravi 
zahtev za promenu. Aktivnost ne mora da bude u okviru GitHub-a. Ona može doći preko poziva od strane 
nekog eksternog sistema.

\subsubsection{Posao}
Posao predstavlja skup koraka koji treba da se izvrše nad istim izvršiocem. Tok rada sa više poslova 
će pokrenuti ove poslove u paraleli. Naravno, tok rada se može podesiti tako da izvršava poslove 
sekvencijalno. Na primer, tok rada može imati dva sekvencijalna posla koji izgrađuju i testiraju kod, 
gde je posao za testiranje zavisan od toga da li kod uopšte može da se izgradi. Ako posao za izgradnju 
ne uspe, posao za testiranje se neće ni pokretati.

\subsubsection{Korak}
Korak predstavlja jednu zadatak koji može pokrenuti komandu unutar posla. Korak može biti ili akcija ili 
komandni skript. Svaki korak unutar posla se izvršava nad istim izvršiocem, što dozvoljava akcijama da 
dele podatke između sebe.

\subsubsection{Akcija}
Akcije predstavljaju samostalne komande koje se mogu kombinovati u korake kako bi sačinile jedan posao.
Akcije su najmanje portabilne jedinice građe toka rada. Korisnici mogu sami da naprave svoje akcije,
ili da koriste akcije koje su napravljene od strane GitHub zajednice. 

\subsubsection{Izvršilac}
Izvršilac je mašina nad kojom se GitHub Akcije izvršavaju. Izvršilac osluškuje pokrenute poslove,
pokreće jedan posao za drugim, i šalje izveštaje GitHub-u o napretku, logovima i rezultatima. Ove 
mašine mogu da budu bazirane na Linux, Windows i macOS operativnim sistemima, a svaki posao 
unutar toka rada se pokreće iz novog virtuelnog okruženja.~\cite{GitHubActions}

\chapter{Slučajevi upotrebe aplikacije Interpres}\label{ch:slucajevi_upotrebe}


\renewcommand{\labelenumii}{\arabic{enumi}.\arabic{enumii}}
\renewcommand{\labelenumiii}{\arabic{enumi}.\arabic{enumii}.\arabic{enumiii}}
\renewcommand{\labelenumiv}{\arabic{enumi}.\arabic{enumii}.\arabic{enumiii}.\arabic{enumiv}}

Razvoj softvera iterativnom metodom diktira i uređivanje prevoda za aplikaciju. Na kraju svakog 
ciklusa, programeri obaveste prevodioce da su završili kako bi prevodioci mogli da započnu svoj 
ciklus prevođenja. 

U ovom poglavlju će biti opisani slučajevi upotrebe. Na dijagramu \ref{fig:slucajevi_upotrebe} je 
prikazana šira slika svih slučajeva upotrebe. 

\begin{figure}[H]
    \centering
    \includegraphics[width=1\textwidth]{slucajevi_upotrebe}
    \caption{BPMN dijagram slučajeva upotrebe}
    \label{fig:slucajevi_upotrebe}
\end{figure}

%%%%%%%%%%%%%%%%%%%%%%
\section{Prijava korisnika uz pomoć GitHub IdP}

\textbf{Kratak opis}: Korisnik se prijavljuje sa sopstvenim kredencijalima

\textbf{Učesnici}: Korisnik

\textbf{Postuslov}: Korisnik je prijavljen

\textbf{Glavni tok}:
\begin{enumerate}
    \item Korisnik klikne na dugme za prijavu
    \item Sistem šalje zahtev prema \textit{Gateway}-u za prijavu
    \item \textit{Gateway} vraća preusmeravanje (302 \textit{Redirect}) na stranicu za prijavu 
    provajdera identiteta (u ovom slučaju \textit{GitHub})
    \item Šalje se zahtev za dobijanje stranice za prijavu provajdera 
    identiteta
    \item \textit{GitHub} šalje stranicu za prijavu korisnika
    \item Korisnik unosi svoje kredencijale
    \item Korisnik šalje zahtev za prijavu klikom na dugme
    \item \textit{GitHub} dobija podatke za prijavu i nakon uspešnog prijavljivanja 
    šalje kod za pristup veb aplikaciji
    \item Veb aplikacija prosleđuje \textit{Gateway}-u kod za pristup
    \item \textit{Gateway}, uz zahtev za dobijanje podataka o korisniku, prosleđuje 
    kod za pristup \textit{GitHub}-u 
    \item \textit{GitHub Gateway}-u vraća podatke o korisniku
    \item Na osnovu dobijenih podataka \textit{Gateway} generiše \textit{JWT} token
    \item \textit{Gateway} šalje \textit{JWT} token veb aplikaciji
    \item Veb aplikacija omogućava prikaz zaštićene stranice korisniku 
    i čuva \textit{JWT} token za buduće komunikacije
\end{enumerate}

\begin{figure}[H]
    \centering
    \includegraphics[width=1\textwidth]{prijava_korisnika}
    \caption{Dijagram sekvence -- prijava korisnika}
\end{figure}


%%%%%%%%%%%%%%%%%%%%%%
\section{Pravljenje projekta}

\textbf{Kratak opis}: Prevodilac ima mogućnost da napravi novi projekat

\textbf{Učesnici}: Prevodilac

\textbf{Postuslov}: Projekat je napravljen

\textbf{Glavni tok}:
\begin{enumerate}
    \item Sistem prikazuje opciju za pravljenje novog projekta
    \item Prevodilac bira akciju za pravljenje novog projekta
    \item Sistem prikazuje formu sa tekstualnim poljima za unos podataka o projektu
    \item Prevodilac unosi 
    \begin{itemize}
        \item naziv projekta
        \item vlasnika \textit{GitHub} repozitorijuma
        \item \textit{GitHub} repozitorijum
        \item putanju na kojoj će se čuvati fajlovi sa prevodima i
        \item skup jezika za koje su potrebni prevodi.
    \end{itemize}
    \item Prevodilac klikne na dugme za testiranje konekcije
    \begin{itemize}
        \item Ukoliko su podešavanja pravilno unešena i \textit{GitHub} aplikacija uspešno instalirana u repozitorijum,
        sistem prikazuje uspešnu poruku zajedno sa jezicima koje je pronašao u repozitorijumu
        \item Ukoliko podešavanja nisu pravilno unešena ili \textit{GitHub} aplikacija nije uspešno instalirana u repozitorijum,
        sistem prikazuje poruku sa greškom.
    \end{itemize}
    \item Prevodilac klikne na dugme za pravljenje projekta
    \item Sistem čuva projekat
    \item Sistem prevodiocu vraća stranicu sa podacima o projektu
\end{enumerate}

\begin{figure}[H]
    \centering
    \includegraphics[width=1\textwidth]{create_project}
    \caption{Korisnički interfejs -- pravljenje projekta}
\end{figure}


%%%%%%%%%%%%%%%%%%%%%%
\section{Uvoz prevoda}

\textbf{Kratak opis}: Prevodilac želi da uveze prevode sa \textit{GitHub}-a

\textbf{Učesnici}: Prevodilac

\textbf{Postuslov}: Prevodilac se nalazi na projektu u koji želi da uveze prevode

\textbf{Glavni tok}:
\begin{enumerate}
    \item Sistem prikazuje opcije za uvoz i izvoz prevoda
    \item Prevodilac bira akciju za uvoz prevoda
    \item Sistem prikazuje modal sa porukom da će se uvozom prevoda sve 
    izmene koje nisu izvezene izgubiti
    \item Prevodilac klikne na dugme "\textit{OK}" ukoliko nema prevoda koji nisu sačuvani
    \item Sistem uvozi prevode
    \item Sistem zatvara modal
\end{enumerate}

\textbf{Alternativni tok}: Ukoliko prevodilac u koraku 4. klikne na dugme "\textit{Cancel}", 
sistem zatvara modal i prevodi neće biti uvezeni.


%%%%%%%%%%%%%%%%%%%%%%
\section{Izmena prevoda}

\textbf{Kratak opis}: Prevodilac vrši izmene kako bi dodao novi prevod ili izmenio postojeći

\textbf{Učesnici}: Prevodilac

\textbf{Postuslov}: Uspešno izvršena izmena prevoda

\textbf{Glavni tok}:
\begin{enumerate}
    \item Sistem prikazuje listu prevoda sa odgovarajućim ključevima
    \item Prevodilac klikne na željeni ključ za koji želi da unese ili izmeni prevod
    \item Sistem vraća tekstualna polja za unos za onoliko jezika koliko je u tom 
    trenutku dostupno za izabrani ključ
    \item Prevodilac unosi ili menja jedan ili više prevoda
    \item Prevod se automatski šalje na čuvanje u trenutku kada tekstualno polje izgubi fokus
    \item Sistem čuva izmenjeni prevod
\end{enumerate}

\begin{figure}[H]
    \centering
    \includegraphics[width=1\textwidth]{translation_editor}
    \caption{Korisnički interfejs uređivača prevoda}
\end{figure}


%%%%%%%%%%%%%%%%%%%%%%
\section{Izvoz prevoda}

\textbf{Kratak opis}: Prevodilac želi da izveze prevode izabranog projekta i 
na taj način napravi zahtev za promenu na \textit{GitHub}-u

\textbf{Učesnici}: Prevodilac

\textbf{Preduslov}: Prevodilac se nalazi na projektu koji želi da izveze

\textbf{Postuslov}: Uspešno izvezen sadržaj prevoda

\textbf{Glavni tok}:
\begin{enumerate}
    \item Sistem prikazuje opcije za uvoz i izvoz prevoda
    \item Prevodilac bira akciju za izvoz prevoda
    \item Sistem prikazuje modal sa formom za unos naslova i opisa 
    zahteva za promenu
    \item Prevodilac unosi naslov i opis u tekstualna polja
    \item Korisnik klikne na dugme \textit{OK}
    \item Sistem pokreće akciju izvoza prevoda
    \item Na \textit{GitHub}-u se pravi zahtev za promenu sa izmenjenim fajlovima za prevode
\end{enumerate}

\begin{figure}[H]
    \centering
    \includegraphics[width=0.7\textwidth]{izvoz_prevoda}
    \caption{Dijagram aktivnosti -- izvoz prevoda}
\end{figure}

\begin{figure}[H]
    \centering
    \includegraphics[width=1\textwidth]{export_translations}
    \caption{Korisnički interfejs za izvoz prevoda}
\end{figure}


\chapter{Implementacija}\label{ch:impl}

Implementacija prati opisanu arhitekturu iz prethodnih poglavlja. 
Korisnički interfejs je zasnovan na arhitekturi jedne stranice, 
uz pomoć okruženja \textit{React}. Komunikacija između klijenta i servera 
je preko \textit{REST API}-a. Server je implementiran u arhitekturi
mikroservisa uz pomoć \textit{NestJS}. Za bazu podataka 
je izabrana \textit{PostgreSQL}. Za potrebe pravljenja zahteva za promenu 
na kodu, server komunicira preko \textit{GitHub} aplikacije. Autentikacija 
je implementirana preko \textit{OAuth} protokola, a za provajdera 
identiteta je izabran \textit{GitHub IdP}. Kao alat za \textit{CI/CD} se koriste 
\textit{GitHub Akcije}. 

Implementirano rešenje je nazvano \textit{Interpres}. Aplikacija je 
višejezična, odnosno može prikazati korisnički interfejs u više 
jezika. Za potrebe uređivanja prevoda je korišćen upravo \textit{Interpres}.

Izvorni kod je otvoren i može se naći na \url{https://github.com/enco164/interpres}.


\section{Komponente sistema}

Na slici \ref{fig:komponente} su prikazane komponente sistema kao i
njihove zavisnosti. Važno je napomenuti da, iako mikroservisi \textit{core} i 
\textit{user-management} koriste različite baze podataka,
obe baze se zapravo nalaze u istom sistemu za upravljanje bazama podataka. 
Ova odluka je donešena radi boljeg iskorišćenja resursa. 

\begin{figure}[h]
  \centering
  \includegraphics[width=0.7\textwidth]{komponente}
  \caption{Komponente sistema}
  \label{fig:komponente}
\end{figure}

U nastavku će biti opisana svaka komponenta, kao i njena 
zavisnost sa drugim komponentama.

\subsection{SPA}
Komponenta \textit{SPA} predstavlja klijentsku stranu aplikacije, implementiranu 
u stilu arhitekture jedne stranicne. Izgrađena je 
uz pomoć razvojnog okruženja \textit{React}. Za stilizovanje korisničkog 
interfejsa korišćena je biblioteka \textit{Material UI}. 

Kada se pokrene izgradnja \textit{React} projekta, artifakti koji se dobiju 
su \textit{HTML}, \textit{JavaScript} i \textit{CSS} fajlovi. Da bi korisnik dobio fajlove, 
potreban je veb server, i u te svrhe je izabran \textit{Nginx}.

Komponenta \textit{SPA} dobija podatke sa servera. Komunikacija sa serverom 
je preko protokola \textit{REST}, a ulazna tačka je mikroservis \textit{gateway}.

\subsection{Gateway}
Komponenta \textit{gateway} jedina "otvara kapiju" ka spoljnom svetu.
Ona služi da prihvati zahteve sa klijentske strane i prosledi ih 
drugim mikroservisima. S obzirom da ona predstavlja svojevrsnu "kapiju",
autentikacija je implementirana baš tu.

Provera pristupa se radi preko \textit{GitHub IdP}. Ako se korisnik 
prvi put prijavljuje na sistem, njegovi podaci o imenu i prezimenu će 
biti poslati mikroservisu \textit{user-management}. 

Komunikacija sa ostalim mikroservisima (\textit{core} i \textit{user-management})
se odvija preko protokola \textit{TCP} podržanog od strane \textit{NestJS}.

\subsection{User Management}
Cilj ove komponente je da se brine o podacima korisnika. Pošto se podaci o korisnicima 
preuzimaju od eksternih servisa, u ovom slučaju od \textit{GitHub}-a, kao dobra praksa se 
pokazalo da ne treba zavisiti od ključeva eksternih sistema. 
Ovde se konkretno misli na jedinstveni identifikator korisnika. Kao 
poboljšanje sistema, mogao bi da se implementira pristup sistemu preko 
nekog drugog provajdera identiteta. U tom slučaju može doći do kolizije 
ključeva, odnosno ne možemo da budemo sigurni da će različiti provajderi identiteta 
davati različite ključeve.

Pored brige o primarnim ključevima za korisnike, ovaj mikroservis služi 
i kao svojevrsna optimizacija. Naime, mnogo je brže kontaktirati 
mikroservis koji je u \textit{Kubernetes} klasteru nego neki eksterni 
servis.

\subsection{Core}
Mikroservis \textit{core} predstavlja srž aplikacije. On se bavi čuvanjem 
podataka o prevodima i o projektima. Tu se nalazi i poslovna logika za 
grupisanje i preslikavanje prevoda u strukturu koja je pogodna za 
klijentsku stranu aplikacije ili za mikroservis \textit{integration}.

\subsection{Integration}
Ova komponenta ima funkciju integracije sa sistemom za verzionisanje koda. 
Na zahtev mikroservisa \textit{core}, ona može preko \textit{GitHub Aplikacije}, 
da dohvati prevode sa repozitorijuma i da napravi zahtev za promenu sa 
novim izmenama. Ako bi se u budućnosti implementirala integracija sa nekim 
drugim sistemom za verzionisanje koda, ovaj mikroservis je pravo mesto za to.

\subsection{GitHub Application i GitHub IdP}
Da bi se napravila integracija sa \textit{GitHub}-om, za potrebe menjanja 
koda, potrebno je napraviti \textit{GitHub Application}. Aplikacija 
ustvari daje samo pristupne ključeve, a na programeru je dalje da implementira 
i postavi aplikaciju na neki server. Implementacija aplikacije u \textit{NodeJS} 
je preko \textit{GitHub}-ove biblioteke, nazvane \textit{octokit}. Preko ove 
biblioteke se dobija interfejs za sve akcije koje je moguće uraditi u 
repozitorijumu. Dve glavne koje su implementirane su čitanje fajlova sa 
prevodima i pravljenje zahteva za promenu, i koriste se za uvoz i za 
izvoz prevoda.

\textit{GitHub IdP} je eksterna komponenta i služi kao provajder identiteta. 
Slično kao i za \textit{GitHub Application}, potrebna je samo registracija 
za dobijanje pristupnih ključeva kojima se pristupa interfejsu provajdera. 
Za implementaciju pristupa korićena je biblioteka \textit{Passport.js}, 
preporučena od strane \textit{NestJS}, koja ima implementaciju 
protokola \textit{OAuth2}, koji koristi \textit{GitHub IdP}.

\section{Klijent}
Klijentski deo je izgrađen uz pomoć radnog okvira \textit{React}, a početna 
organizacija koda uz pomoć \textit{create-react-app}, preporučenog alata za 
generisanje \textit{React} projekta od strane \textit{Facebook}-a. 
Projekat je izgenerisan na jeziku \textit{Typescript}. 

Organizacija projekata je podeljena po funkcionalnostima, i one su 
\textit{auth} (odgovorna za autorizaciju), \textit{projects} (odgovorna za 
podešavanje projekta), \textit{import-export} (odgovorna za 
uvoz, odnosno izvoz prevoda) i \textit{translations} (odgovorna za uređivanje 
prevoda).

Sve \textit{React} komponente su pisane kao funkcijske komponente 
(eng. \textit{function components}) uz korišćenje kuka (eng. \textit{hooks}).
Programeri koji po prvi put koriste \textit{React} uglavnom nalaze da je 
ovako napisan kod nečitljiv jer deluje da je pomešana poslovna logika komponente 
sa prikazom korisničkog interfejsa. Ustvari, mišljenje da je poslovna logika 
pomešana sa prikazom korisničkog interfejsa nastaje iz toga da u zvaničnoj 
dokumentaciji nije predložena arhitektura, već se samo opisuje tehnologija.
Programeru je ostavljeno na razmišljanje kako da organizuje svoju aplikaciju.
Prilikom razvoja projekta poštovan je princip da se sva poslovna logika 
nalazi u kuki, a da funkcijska komponenta koristi kuku i prikazuje sadržaj. 
Ovim principom se postiže veća čitljivost, razdvajaju se odgovornosti, lakše se testira 
automatskim testovima, a samim tim se i kod lakše održava. 
Ovaj princip je prikazan na primeru koda \ref{code:primer}.

\begin{listing}[h]
  \centering
  \begin{minted}[ fontsize=\footnotesize ]{jsx}
  // kuka: usePrimer.ts 
  export const usePrimer = () => {
    const [count, setCount] = useState(0);

    return {
      count,
      handleClick: () => setCount(count + 1),
    };
  }

  // komponenta: primer.tsx 
  export const Primer = () => {
    const {count, handleClick} = usePrimer();

    return (
      <div>
        <p>Kliknuli ste {count} puta</p>
        <button onClick={handleClick}>
          Klikni me
        </button>
      </div>
    );
  }
  \end{minted}
\caption{Princip pisanja funkcijskih komponenata sa kukama}
\label{code:primer}
\end{listing}

Za upravljanje stanjem u aplikaciji korišćen je \textit{redux}. On radi 
po principu centralizovanog skladišta, što znači da njemu pristupaju sve 
komponente. Na taj način je napravljena sigurnost da će podaci biti konzistentni 
i da neće biti particionisani. Ako dve komponente zahtevaju isti podatak, one ga neće čuvati u svom stanju 
već će ga potraživati sa istog mesta.

\section{Server}
Svaki mikroservis je napravljen kao zasebna aplikacija i generisan uz pomoć 
\textit{NestJS} interfejsa za komandnu liniju. Ako neki mikroservis treba 
da zna za postojanje nekog drugog, informacija o lokaciji će biti prosleđenja 
kroz konfiguraciju, odnosno kroz sistemske promenljive. U konfiguraciji 
se pored toga čuvaju pristupni ključevi za \textit{GitHub Application}, 
\textit{GiHub IdP}, i lokacija i kredencijali za bazu podataka. Konfiguracija 
mikroservisa se učitava pri svakom podizanju.

Mikroservisi koji čuvaju stanje u bazi podataka su \textit{core} i 
\textit{user-management}. Kako \textit{NestJS} koristi u pozadini \textit{TypeORM}, 
iskorišćena je njegova funkcija migriranja baze podataka. Migracije služe 
za promenu sheme baze podataka u produkcionom okruženju. One osiguravaju da 
će prebacivanje na novu verziju sheme biti sigurno i da neće doći do gubitka 
podataka. Migracioni fajl sadrži klasu koja implementira 
\texttt{MigrationInterface}, a potrebno je implementirati dve metode: 
\texttt{up} i \texttt{down}. Prva služi da se baza migrira na višu 
verziju, a druga služi ako je u nekom slučaju potreban povratak na prethodnu. 
Unutar tih metoda treba napisati \textit{SQL} naredbe za migracije. Pored 
ručnog pisanja migracija, \textit{TypeORM} pruža i mogućnost generisanja 
migracionih klasa, jer može izračunati prethodni oblik modela, a odatle i 
razliku koju treba primeniti na bazu podataka kako bi podržala novi model. 

Stil pisanja koda na serveru je reaktivan uz \textit{rx.js} biblioteku koja
implementira obrazac "posmatrač". Server napisan u stilu mikroservisa,
a komunikacija među servisima je sinhrona, odnosno po principu 
"zahtev -- odgovor". To znači da dok neki servis čeka na odgovor drugog 
servisa, prvi ostaje blokiran dok ne dobije odgorvor. Kako se programi napisani 
u \textit{JavaScript}-u izvršavaju u jednoj niti ovo postaje veliki problem. 
Iz tog razloga \textit{NestJS} pruža implementaciju \textit{HTTP} klijenta 
koji prima odgovore asinhrono. On će poslati \textit{HTTP} zahtev koji će biti razrešen kada 
stigne odgovor. Na taj način mikroservis koji je poslao \textit{HTTP} zahtev 
može obavljati i neki drugi posao dok odgovor ne stigne. Korišćenjem obrasca 
posmatrač se ova asinhronost lakše apstrahuje. Potrebno je napraviti zahtev 
i onda se pretplatiti na odgovor. Dok se čeka odgovor, pogram je slobodan da 
obavlja neki drugi posao. Kada odgovor stigne biće pozvana funkcija 
za obradu pretplate i tok izvršavanja će se nastaviti.
Pisanje reaktivnog koda je umnogome olakšano jer je i sam \textit{NestJS} 
napisan uz pomoć \textit{rx.js}. Na primeru koda \ref{code:rxjs} je prikazano
korišćenje \textit{rx.js} na metodi uvoza prevoda.


\begin{listing}[h]
  \centering
  \begin{minted}[ fontsize=\footnotesize ]{js}
importProject({ projectId }: ImportRequest) {
  return this.projectRepository.getProjectById(projectId)
    .pipe(
      throwIfEmpty(
        () => new NotFoundException(`Project with id ${projectId} not found`)
      ),
      concatMap((project) =>
        forkJoin([
          this.integrationMicroserviceClientSend(
            "import",
            {
              owner: project.githubOwner,
              repo: project.githubRepo,
              translationsLoadPath: project.lngLoadPath,
            }
          ),
          this.translationRepository.remove(project.translations)
        ])
      ),
      concatMap(([dataFromGithub, ]) =>
        this.importParsedTranslations(dataFromGithub, projectId)
      )
    );
}  
  \end{minted}
\caption{Metoda za uvoz prevoda napisana u reaktivnom stilu}
\label{code:rxjs}
\end{listing}


\section{Automatizacija}
Kontinualna integracija, isporučivanje i raspoređivanje je implementirano uz 
pomoć \textit{GitHub}-a. Na glavnoj (\textit{"main"}) grani 
je postavljeno pravilo zaštite, odnosno onemogućeno je direkto slanje 
koda na tu granu. Za svaku izmenu koda potrebno otvoriti zahtev za 
promenu. Pored toga, postavljeno je pravilo da, grana koju treba spojiti 
na glavnu granu, mora sadržati sve izmene koje se nalaze na glavnoj grani.

Na događaj otvaranja novog zahteva za izmenu koda, pokreće se niz 
\textit{GitHub} akcija. Za klijentski deo i za svaki mikroservis 
se pokreće naredba izgradnje i naredba testiranja. Uz navedena 
ograničenja za spajanje grana, i sa ovom \textit{GitHub} akcijom, 
osigurava se da će spojeni kod biti istestiran pre spajanja. To znači da 
na glavnoj grani ne bi trebalo da se pojavi neka greška koja bi inače mogla 
da se otkrije testiranjem.

Kao i za zahtev za izmenu koda, kada se grana spoji u glavnu granu 
postoji niz \textit{GitHub} akcija, za svaku komponentu po jedna akcija.
Tu se izgrade \textit{Docker} slike koje se kasnije isporučuju na 
\textit{Docker} javni registar. Potom, kada su sve slike isporučene na 
\textit{Docker} registar, pokreće se \textit{GitHub} akcija koja započinje 
raspoređivanje na \textit{Kubernetes}.

Na primeru koda \ref{code:cicd} je prikazan tok kontinualne isporuke
mikroservisa \textit{gateway} na \textit{Docker} javni registar.

\begin{listing}[h]
  \centering
  \begin{minted}[ fontsize=\footnotesize ]{yaml}
name: Build & Publish API Gateway docker image

on:
  push:
    branches:
      - main

jobs:
  build-docker-and-publish:
    runs-on: ubuntu-latest
    environment: main

    steps:
      - name: Checkout code
        uses: actions/checkout@v2

      - name: Use Node.js 14
        uses: actions/setup-node@v1
        with:
          node-version: '14'

      - name: Determine Docker Tag
        run: echo "DOCKER_TAG=${{ github.sha }}"  >> $GITHUB_ENV
      - run: |-
          echo DOCKER_TAG:  $DOCKER_TAG

      - name: DockerHub login
        env:
          DOCKERHUB_USERNAME: ${{ secrets.DOCKERHUB_USERNAME }}
          DOCKERHUB_PASSWORD: ${{ secrets.DOCKERHUB_PASSWORD }}
        run: |
          docker login -u $DOCKERHUB_USERNAME -p $DOCKERHUB_PASSWORD

      - name: Build docker image
        run: |
          docker build \
            -t interpres/api-gateway:latest \
            -t interpres/api-gateway:$DOCKER_TAG \
            ./server/api-gateway

      - name: Push docker image
        run: |
          docker push --all-tags interpres/api-gateway
\end{minted}
\caption{Tok kontinualne isporuke mikroservisa \textit{gateway}}
\label{code:cicd}
\end{listing}

\section{Produkciono okruženje}
Razne kompanije pružaju usluge iznajmljivanja računara u oblaku. Neki od 
poznatijih proizvoda su \textit{Amazon Web Services}, \textit{Microsoft Azure}
i \textit{Google Cloud Platform}. Sa druge strane, \textit{Kubernetes} je 
nezavisan od platforme. Iako ga je moguće instalirati i pokrenuti na sopstvenom 
serveru, taj posao je mukotrpan pa je pametnije izabrati neki proizvod gde se 
\textit{Kubernetes} može podići uz par klikova. 

Kako je \textit{Google} razvio \textit{Kubernetes}, pretpostavka je da je 
na \textit{Google Cloud Platform} uvek malo prednjači, pa je iz tog razloga 
on izabran za aplikaciju \textit{Interpres}.

Podešavanje na \textit{Google Cloud Platform} je jednostavno. Preko 
korisničkog interfejsa je potrebno napraviti klaster. Za klaster je 
potrebno izabrati tip virtualne mašine, kao i lokaciju servera na kojoj 
će ta virtualna mašina biti podignuta. Nadalje se sve može konfigurisati 
i preko komandne linije uz \textit{Kubernetes}-ov alat \texttt{kubectl}.
Preko ovog alata se izvršavaju komande sa pravljenje čaura.

Potrebno je napomenuti da se korišćenje \textit{Google Cloud Platform}-e,
naravno, naplaćuje. U trenutku pisanja ovog rada \textit{Google} za nove 
korisnike obezbeđuje besplatnih 300\$, koji su dovoljni za testiranje.
Svakako treba voditi računa prilikom podešavanja kako ne bi došlo do 
nepotrebnih troškova.


\chapter{Zaključak}\label{ch:zakljucak}

Cilj ovog rada je da predloži i obrazloži informacioni sistem u oblaku koji bi pomogao 
programerima i prevodiocima tokom razvijanja višejezične aplikacije. Korišćenjem ovog 
sistema smanjuje se jaz između ove dve grupe koje zajedničim snagama poboljšavaju 
korisničko iskustvo aplikacije na kojoj rade. Zbog sve prisutnije digitalizacije, 
ne sme se zaboraviti na grupe ljudi koje ne razumeju određeni jezik i zbog toga 
ne mogu da koriste aplikaciju.

Prilikom razvoja sistema korišćena je arhitektura mikroservisa. Ideja vodilja za odabir 
ove arhitekture je bila da informacioni sistem bude što je više moguće skalabilan, 
pouzdan i održiv. Sve veći trend korišćenja mikroservisa u oblaku za velike aplikacije 
potvrđuje da je to dobar izbor. 

Kako se računarstvo u oblaku razvija velikom brzinom, tako se razvijaju i novi alati 
koji pokušavaju da reše probleme koji nastaju na oblaku. S obzirom na sve veću upotrebu 
arhitekture mikroservisa, glavni problem predstavlja upravljanje većim brojem komponenti.
\textit{Kubernetes}, kao alat za rešavanje ovog problema, u potpunosti zadovoljava 
potrebe operacionog tima, ali i tima programera.

Alati za automatizaciju u mnogome pomažu razvoj softvera. Kontinuirano isporučivanje 
i kontinuirano raspoređivanje ubrzavaju ciklus razvoja i korisnici ranije mogu dobiti 
nove verzije softvera. Pored samih korisnika, benefit imaju i programeri jer mogu 
ranije da uoče probleme i brže da ih reše.

Potrebno je napomenuti da mikroservisi ne predstavljaju rešenje za sve. Iako se takvi 
sistemi lakše skaliraju i imaju bolju razdvojenost poslovne logike, sam razvoj ume da 
bude kompleksan. Prerano razmišljanje o optimizaciji i skaliranju sistema može usporiti 
razvoj i potencijalno dovesti do gašenja projekta jer funkcionalnosti sistema nisu 
isporučene na vreme. Sa druge strane, može se desiti da sam sistem uopšte nema potrebe 
da opsluži toliko korisnika. Onda se dobija kompleksan sistem bez prevelikog iskorišćenja,
što je potpuno suprotno od polazne pretpostavke. Potrebno je voditi se principom da se 
sistem optimizuje samo onda kada je to i zaista potrebno.


% ------------------------------------------------------------------------------
% Literatura
% ------------------------------------------------------------------------------
\literatura

% ==============================================================================
% Završni deo teze i prilozi
\backmatter
% ==============================================================================

% ------------------------------------------------------------------------------
% Biografija kandidata
%\begin{biografija}
%  \textbf{Vuk Stefanović Karadžić} (\emph{Tršić,
%    26. oktobar/6. novembar 1787. — Beč, 7. februar 1864.}) bio je
%  srpski filolog, reformator srpskog jezika, sakupljač narodnih
%  umotvorina i pisac prvog rečnika srpskog jezika.  Vuk je
%  najznačajnija ličnost srpske književnosti prve polovine XIX
%  veka. Stekao je i nekoliko počasnih mastera.  Učestvovao je u
%  Prvom srpskom ustanku kao pisar i činovnik u Negotinskoj krajini, a
%  nakon sloma ustanka preselio se u Beč, 1813. godine. Tu je upoznao
%  Jerneja Kopitara, cenzora slovenskih knjiga, na čiji je podsticaj
%  krenuo u prikupljanje srpskih narodnih pesama, reformu ćirilice i
%  borbu za uvođenje narodnog jezika u srpsku književnost. Vukovim
%  reformama u srpski jezik je uveden fonetski pravopis, a srpski jezik
%  je potisnuo slavenosrpski jezik koji je u to vreme bio jezik
%  obrazovanih ljudi. Tako se kao najvažnije godine Vukove reforme
%  ističu 1818., 1836., 1839., 1847. i 1852.
%\end{biografija}
% ------------------------------------------------------------------------------

\end{document}

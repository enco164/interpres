% Format teze zasnovan je na paketu memoir
% http://tug.ctan.org/macros/latex/contrib/memoir/memman.pdf ili
% http://texdoc.net/texmf-dist/doc/latex/memoir/memman.pdf
%
% Prilikom zadavanja klase memoir, navedenim opcijama se podešava
% veličina slova (12pt) i jednostrano štampanje (oneside).
% Ove parametre možete menjati samo ako pravite nezvanične verzije
% mastera za privatnu upotrebu (na primer, u b5 varijanti ima smisla
% smanjiti
\documentclass[12pt,oneside]{memoir}

% Paket koji definiše sve specifičnosti master rada Matematičkog fakulteta
\usepackage[latinica]{matfmaster}
%
% Podrazumevano pismo je ćirilica.
%   Ako koristite pdflatex, a ne xetex, sav latinički tekst na srpskom jeziku
%   treba biti okružen sa \lat{...} ili \begin{latinica}...\end{latinica}.
%
% Opicija [latinica]:
%   ako želite da pišete latiniciom, dodajte opciju "latinica" tj.
%   prethodni paket uključite pomoću: \usepackage[latinica]{matfmaster}.
%   Ako koristite pdflatex, a ne xetex, sav ćirilički tekst treba biti
%   okružen sa \cir{...} ili \begin{cirilica}...\end{cirilica}.
%
% Opcija [biblatex]:
%   ako želite da koristite reference na više jezika i umesto paketa
%   bibtex da koristite BibLaTeX/Biber, dodajte opciju "biblatex" tj.
%   prethodni paket uključite pomoću: \usepackage[biblatex]{matfmaster}
%
% Opcija [b5paper]:
%   ako želite da napravite verziju teze u manjem (b5) formatu, navedite
%   opciju "b5paper", tj. prethodni paket uključite pomoću:
%   \usepackage[b5paper]{matfmaster}. Tada ima smisla razmisliti o promeni
%   veličine slova (izmenom opcije 12pt na 11pt u \documentclass{memoir}).
%
% Naravno, opcije je moguće kombinovati.
% Npr. \usepackage[b5paper,biblatex]{matfmaster}

% Pomoćni paket koji generiše nasumičan tekst u kojem se javljaju sva slova
% azbuke (nema potrebe koristiti ovo u pravim disertacijama)
%\usepackage[latinica]{pangrami}

% Datoteka sa literaturom u BibTex tj. BibLaTeX/Biber formatu
\bib{matfmaster}

% Ime kandidata na srpskom jeziku (u odabranom pismu)
\autor{Uroš Milenković}
% Naslov teze na srpskom jeziku (u odabranom pismu)
\naslov{Razvoj veb aplikacije u oblaku za upravljanje prevodima u višejezičnim aplikacijama}
% Godina u kojoj je teza predana komisiji
\godina{2021}
% Ime i afilijacija mentora (u odabranom pismu)
\mentor{prof. dr Saša \textsc{Malkov}, vanredni profesor\\ Univerzitet u Beogradu, Matematički fakultet}
% Ime i afilijacija prvog člana komisije (u odabranom pismu)
\komisijaA{Anđelka \textsc{Zečević}, asistent\\ Univerzitet u Beogradu, Matematički fakultet}
% Ime i afilijacija drugog člana komisije (u odabranom pismu)
\komisijaB{Anđelka \textsc{Zečević}, asistent\\ Univerzitet u Beogradu, Matematički fakultet}
% Ime i afilijacija trećeg člana komisije (opciono)
% \komisijaC{}
% Ime i afilijacija četvrtog člana komisije (opciono)
% \komisijaD{}
% Datum odbrane (odkomentarisati narednu liniju i upisati datum odbrane ako je poznat)
% \datumodbrane{}

% Apstrakt na srpskom jeziku (u odabranom pismu)
\apstr{%
Apstrakt
}

% Ključne reči na srpskom jeziku (u odabranom pismu)
\kljucnereci{računarstvo u oblaku, višejezičnost}

\begin{document}
% ==============================================================================
% Uvodni deo teze
\frontmatter
% ==============================================================================
% Naslovna strana
\naslovna
% Strana sa podacima o mentoru i članovima komisije
\komisija
% Strana sa posvetom (u odabranom pismu)
\posveta{Mami, tati i dedi}
% Strana sa podacima o disertaciji na srpskom jeziku
\apstrakt
% Sadržaj teze
\tableofcontents*

% ==============================================================================
% Glavni deo teze
\mainmatter
% ==============================================================================
\chapter{Uvod}\label{ch:uvod}

Aplikacije koje su namenjene za globalno tržište treba pripremiti 
tako da korisnici koji pripadaju različitim govornim područijima i 
geografskim regionima razumeju sadržaj. Postupak dizajniranja aplikacije 
tako da podrži različite jezike se naziva internacionalizacija, a osobina 
aplikacije koja podržava više jezika se naziva višejezičnost. 

Proces prilagođavanja softvera za različite jezike može biti kompleksan. 
Veliku pomoć u tom procesu danas pružaju različita rešenja. Odabir rešenja 
uglavnom zavisi od odabira programskog jezika u kom će softver biti 
razvijan, kao i od korišćenja programskog okvira.

Sa druge strane, ne može se očekivati da programer zna sve jezike za koje je 
softver namenjen. To znači da se u proces internacionalizacije uključuju i 
prevodioci koji poznaju jezik na koji softver treba prevesti. Oni implicitno 
postaju osobe koje razvijaju softver.

Pomenuta rešenja za olakšavanje internacionalizacije uglavnom su fokusirana
samo na tehničke probleme, odnosno probleme programera. Prevodi se čuvaju u
fajlovima i obično predstavljaju serijalizovanu strukturu mape, odnosno parove
"ključ -- vrednost". Takvi fajlovi su neretko teški za korišćenje od strane
netehnikičih lica, a prevodioci često nisu tehnička lica. Kako bi što bolje 
obavljali svoj posao, prevodiocima je potrebna neka vrsta alata koja im deluje
poznato, nešto na šta su navikli, odnosno nešto što koriste svaki dan. 
Prevodioce ne zanima koje rešenje su programeri izabrali, niti koji format 
serijalizacije se koristi. Njima je potreban familijaran interfejs za ažuriranje
tih fajlova.

Razvojem računarstva u oblaku, ljudi su navikli da im se sve nalazi u oblaku, 
odnosno da im je sve uvek dostupno, na svakom mestu i u bilo koje vreme. 
Paralelno sa rezvojem softvera u oblaku, popularnost stiču arhitekture zasnovane
na mikroservisima ali i razvoj aplikacija u kontejnerima. Za kontrolisanje 
velikog broja kontejnera se sve više korsti alat \textit{Kubernetes}. Pored toga, 
tendencija je da se hardver, ondnosno infrastruktura, više ne održava u okviru
organizacije, već da se on iznajmljuje od dobavljača računarstva u oblaku 
(eng. \textit{cloud computing provider}). Ovakvim pristupom se omogućava lako 
prenošenje aplikacije sa jednog okruženja na drugo. Razvoj takvog softvera nosi
sa sobom dodatne izazove. 

U ovom radu biće opisan razvoj softvera u oblaku. Fokus će biti stavljen na 
arhitekturu mikroservisa, korišćenje alata \textit{Kubernetes} i proces pravljenja 
višejezične aplikacije. Za razvoj klijentskog, ali i serverskog dela 
kôda, koristiće se programski jezik \textit{JavaScript}. U svrhu ilustracije i
boljeg razumevanja biće razvijena aplikacija za upravljanje prevodima. 
Aplikacija je nazvana Interpres i biće razvijana kao softver otvorenog kôda.

\chapter{Razrada}\label{ch:razrada}
Razrada

\chapter{Zaključak}\label{ch:zakljucak}

Cilj ovog rada je da predloži i obrazloži informacioni sistem u oblaku koji bi pomogao 
programerima i prevodiocima tokom razvijanja višejezične aplikacije. Korišćenjem ovog 
sistema smanjuje se jaz između ove dve grupe koje zajedničim snagama poboljšavaju 
korisničko iskustvo aplikacije na kojoj rade. Zbog sve prisutnije digitalizacije, 
ne sme se zaboraviti na grupe ljudi koje ne razumeju određeni jezik i zbog toga 
ne mogu da koriste aplikaciju.

Prilikom razvoja sistema korišćena je arhitektura mikroservisa. Ideja vodilja za odabir 
ove arhitekture je bila da informacioni sistem bude što je više moguće skalabilan, 
pouzdan i održiv. Sve veći trend korišćenja mikroservisa u oblaku za velike aplikacije 
potvrđuje da je to dobar izbor. 

Kako se računarstvo u oblaku razvija velikom brzinom, tako se razvijaju i novi alati 
koji pokušavaju da reše probleme koji nastaju na oblaku. S obzirom na sve veću upotrebu 
arhitekture mikroservisa, glavni problem predstavlja upravljanje većim brojem komponenti.
\textit{Kubernetes}, kao alat za rešavanje ovog problema, u potpunosti zadovoljava 
potrebe operacionog tima, ali i tima programera.

Alati za automatizaciju u mnogome pomažu razvoj softvera. Kontinuirano isporučivanje 
i kontinuirano raspoređivanje ubrzavaju ciklus razvoja i korisnici ranije mogu dobiti 
nove verzije softvera. Pored samih korisnika, benefit imaju i programeri jer mogu 
ranije da uoče probleme i brže da ih reše.

Potrebno je napomenuti da mikroservisi ne predstavljaju rešenje za sve. Iako se takvi 
sistemi lakše skaliraju i imaju bolju razdvojenost poslovne logike, sam razvoj ume da 
bude kompleksan. Prerano razmišljanje o optimizaciji i skaliranju sistema može usporiti 
razvoj i potencijalno dovesti do gašenja projekta jer funkcionalnosti sistema nisu 
isporučene na vreme. Sa druge strane, može se desiti da sam sistem uopšte nema potrebe 
da opsluži toliko korisnika. Onda se dobija kompleksan sistem bez prevelikog iskorišćenja,
što je potpuno suprotno od polazne pretpostavke. Potrebno je voditi se principom da se 
sistem optimizuje samo onda kada je to i zaista potrebno.


% ------------------------------------------------------------------------------
% Literatura
% ------------------------------------------------------------------------------
\literatura

% ==============================================================================
% Završni deo teze i prilozi
\backmatter
% ==============================================================================

% ------------------------------------------------------------------------------
% Biografija kandidata
%\begin{biografija}
%  \textbf{Vuk Stefanović Karadžić} (\emph{Tršić,
%    26. oktobar/6. novembar 1787. — Beč, 7. februar 1864.}) bio je
%  srpski filolog, reformator srpskog jezika, sakupljač narodnih
%  umotvorina i pisac prvog rečnika srpskog jezika.  Vuk je
%  najznačajnija ličnost srpske književnosti prve polovine XIX
%  veka. Stekao je i nekoliko počasnih mastera.  Učestvovao je u
%  Prvom srpskom ustanku kao pisar i činovnik u Negotinskoj krajini, a
%  nakon sloma ustanka preselio se u Beč, 1813. godine. Tu je upoznao
%  Jerneja Kopitara, cenzora slovenskih knjiga, na čiji je podsticaj
%  krenuo u prikupljanje srpskih narodnih pesama, reformu ćirilice i
%  borbu za uvođenje narodnog jezika u srpsku književnost. Vukovim
%  reformama u srpski jezik je uveden fonetski pravopis, a srpski jezik
%  je potisnuo slavenosrpski jezik koji je u to vreme bio jezik
%  obrazovanih ljudi. Tako se kao najvažnije godine Vukove reforme
%  ističu 1818., 1836., 1839., 1847. i 1852.
%\end{biografija}
% ------------------------------------------------------------------------------

\end{document}
